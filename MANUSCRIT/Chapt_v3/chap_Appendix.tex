%%%%%%%%%%%%%%%%%%don't forget if needed %%%%%%%%%%%%%%%%%%%%%
%\section[toc version]{title version%
%              \sectionmark{head version}}
%\sectionmark{head version}
%%%%%%%%%%%%%%%%%%%%%%%%%%%%%%%%%%%%%%%%%%%%%%%%%%%%%%%%%%%%%%
\def\titcourt{ }
\def\titlong{ }
%%%%%%%%%%%%%%%%%%%%%%%%%%%%%%%%%%%%%%%%%%%%%%%%%%%%%%%%%%%%%%%%
%\chapter[\titlong]{\titlong%
%              \chaptermark{\titcourt}}
%\chaptermark{\titcourt}
%\label{chap-MELTING-CAVITY}


\chapter{Appendix} \label{chap-Appendix}
%%%%%%%%%%%%%%%%%%%%%%%%%%%%%%%%%%%%%%%%%%%%%%%%%%%%%%%%%%%%%%%%
%%%%%%%%%%%%%%%%%%%%%%%%%%%%%%%%%%%%%%%%%%%%%%%%%%%%%%%%%%%%%%%%

\section{Boundary layer approximation and scale analysis} \label{sec-bound-scal-anal}
Either PCM is used for energy storage or for building insulation or for other purposes, one would necessarily assess the heat transfer during the phase-change process.
Mainly, it was extensively proven that the convective heat transfer plays significant role during the melting stage.
Therefore, before solving numerically eqs. (\ref{eq-qmvt}) - (\ref{eq-energ}), we first rely on scale analysis to predict theoretically the fluid flow and heat transfer patterns that can develop in the fluid part.
The idea behind the scaling analysis is about identifying the proper scales of the phenomenon, in order to understand the evolution of the heat transfer and the melting rates.

In the present analysis, we will consider first only the liquid phase without phase-change.
A further analysis of the scale during the melting will be developed in sec. (\ref{chap-MELTING-CAVITY}).
We consider a two-dimensional enclosure of height H filled with Newtonian fluid, differentially heated from the vertical walls and insulated from the horizontal walls.
A No-slip boundary condition is considered for the velocity. 
Since no external force is applied to our system, the fluid flow is merely driven by natural convection flow, induced by temperature differences from the vertical walls.
It is well-known from the foregoing boundary conditions that the fluid layer situated close to the vertical walls stuck to the wall and are motionless.
The heat transfer through the fluid layer immediately adjacent to the wall is accordingly by pure conduction, i.e, $ q = - \left. (\partial \theta/ \partial x) \right |_{x=0} $.
We therefore define the average Nusselt number to quantify the heat transfer rate at the heated wall:
\begin{equation}\label{eq-def-Nu}
   N\!u = - \int_0^1 \left. \frac{\partial \theta}{\partial x} \right |_{x=0} dy.
\end{equation}

When a steady state could be reached, the fluid near each sidewall is characterized by two boundary layers: a thermal boundary layer of thickness $\delta_{\theta}$ and a viscous boundary layer of thickness $\delta_\nu$.
The boundary layer approximation assumes that the flow and the energy transfer are restricted predominantly to the boundary layer region.
This theory was proposed first by Prandtl in 1904 and validated later by many experimental and numerical studies.
The main consequences of the boundary layer approximations are that: \\
{\it (i)} the normal part of the momentum equation has a negligible importance, \\
{\it (ii)} the downstream diffusion term in the momentum and energy equations are neglected in comparison with the normal diffusion terms ($\partial^2 \vec{u}/\partial y^2 \ll \partial^2 \vec{u}/\partial x^2$ and $\partial^2 T/\partial y^2 \ll \partial^2 T/\partial x^2$) since the boundary layer thickness is much smaller than the enclosure height ($ \delta \ll H$), \\
{\it (iii)} the pressure distribution is purely hydrostatic, i.e, $P = - \rho g y$, \\
{\it (iv)} the thermal and the viscous boundary layer thickness are given by the order of magnitude expressions: $\delta_\nu/\delta_\theta = o \left(\Pr^{1/2} \right)$.\\
These assumptions lead to the following boundary layer equations for the conservation of mass, momentum and energy:
\begin{eqnarray} \label{eq-bound-mass}
	\frac{\partial u}{\partial x} + \frac{\partial v}{\partial y} &=& 0, \\  \label{eq-bound-mom}
	u \frac{\partial v}{\partial x} + v \frac{\partial v}{\partial y} &=& \nu \frac{\partial^2 v}{\partial x^2} + g \beta (T - T_{ref}), \\ \label{eq-bound-energy}
	u \frac{\partial T}{\partial x} + v \frac{\partial T}{\partial y} &=& \alpha \frac{\partial^2 T}{\partial x^2}. 
\end{eqnarray}
The mass conservation in eq. (\ref{eq-bound-mass}) in the boundary layer region leads to:
\begin{equation}
	\frac{u}{\delta_\theta} \sim \frac{v}{H}.
\end{equation}
The energy eq. (\ref{eq-bound-energy}) expresses a balance between longitudinal convection and transverse conduction:
\begin{equation}
	\frac{v}{H} \sim \frac{\alpha}{\delta_\theta^2},
\end{equation}
which yields:
\begin{equation} \label{eq-scale-v}
	v \sim \frac{\alpha H}{\delta_\theta^2}.
\end{equation}
As far as momentum eq. (\ref{eq-bound-mom}) is concerned, we could identify the interplay among three forces:
\begin{equation} \label{eq-scale-momentum}
	\underbrace{\frac{v^2}{H}}_{inertia} \quad \underbrace{\nu \frac{v}{\delta_\theta^2}}_{friction} \quad \underbrace{g \beta \delta T}_{buoyancy}.
\end{equation}
Using the expression of $v$ in eq. (\ref{eq-scale-v}) and by dividing eq. (\ref{eq-scale-momentum}) by $g \beta \delta T$, we obtain:
\begin{equation} \label{eq-final-scale}
	\underbrace{\left( \frac{H}{\delta_\theta} \right)^4 Pr ^{-1} \Ray^{-1}}_{inertia} \quad  \underbrace{\left( \frac{H}{\delta_\theta} \right)^4 \Ray^{-1}}_{friction} \quad \underbrace{1}_{buoyancy}.
\end{equation}
Eq. (\ref{eq-final-scale}) indicates that the behaviour of the fluid in the boundary layer depends on the $\Pr$ number. \\
First, for a high-Prandtl fluid ($\Pr \geq 1$), the friction-buoyancy balance yields: 
\begin{equation} \label{eq-scale-nbd-high-Pr}
	\delta_\theta \sim H \Ray^{-1/4}.
\end{equation}
The Nusselt number from eq. (\ref{eq-def-Nu}) scales as $H/\delta_\theta$, resulting: 
\begin{equation}
	N\!u \sim Ra^{1/4}
\end{equation}
and $v \sim \alpha/H \Ray^{1/2}$. \\
Second, for a low-Prandtl fluid ($\Pr \ll 1$), we observe a balance between inertia and buoyancy, leading to:
\begin{equation} \label{eq-corr-Low-Pr}
	\delta_\theta \sim H Pr^{-1/4} \Ray^{-1/4}.
\end{equation}
Accordingly we obtain a Nu-Ra correlation:
\begin{equation}
	Nu \sim Pr^{1/4} \Ray^{1/4}.
\end{equation}

