%%%%%%%%%%%%%%%%%%don't forget if needed %%%%%%%%%%%%%%%%%%%%%
%\section[toc version]{title version%
%              \sectionmark{head version}}
%\sectionmark{head version}
%%%%%%%%%%%%%%%%%%%%%%%%%%%%%%%%%%%%%%%%%%%%%%%%%%%%%%%%%%%%%%
\def\titcourt{Finite element method}
\def\titlong{Finite element method for the Navier-Stokes-Boussinesq equations using Newton algorithm}
%%%%%%%%%%%%%%%%%%%%%%%%%%%%%%%%%%%%%%%%%%%%%%%%%%%%%%%%%%%%%%%%
\chapter[\titlong]{\titlong%
              \chaptermark{\titcourt}}
\chaptermark{\titcourt}
\label{chap-FEM}
%%%%%%%%%%%%%%%%%%%%%%%%%%%%%%%%%%%%%%%%%%%%%%%%%%%%%%%%%%%%%%%%
%%%%%%%%%%%%%%%%%%%%%%%%%%%%%%%%%%%%%%%%%%%%%%%%%%%%%%%%%%%%%%%%

%\section{Numerical method}\label{sec: num meth}

This chapter sets the numerical algorithm for solving the system of equation described previously in chapter \ref{chap-NSB} for both two and three-dimensional configurations.
The finite element method and the Newton algorithm are presented first in detail.
%The accuracy of the developed method is then assessed with some basic theoretical test for two-dimensional configurations.
Then, the domain decomposition method for the three-dimensional configuration is presented.% with a strong scalability analysis.

\section{Motivation for the choice of the numerical method}
Aside from the non-linear convection terms in the momentum and the energy eqs. (\ref{eq-qmvt})-(\ref{eq-energ}), a further difficulty arise from the non-linearity introduced by the source term $\partial (CS)/\partial t$ in eq. (\ref{eq-energ}).
In many in-house or commercial codes used to simulate phase-change problems (ANSYS CFX, Fluent, etc.),
finite difference (FD) or finite volume (FV) methods are used in most of the time with a fixed-mesh approach.
Accurate simulations are therefore carried out by increasing considerably the mesh resolution in the whole domain, increasing consequently dramatically the computational time.
At the example of \cite{voller1996cyclic} who used insufficient grid resolution to compute the melting of gallium, \cite{wang2010numerical} had to consider thinner mesh resulting by increasing by a factor of three the total mesh number to capture correctly the boundary layer structures. 

Our choice for the finite element (FE) methods is motivated by its capability to adapt dynamically the mesh and to deal with several geometrical domain.
The adaptive capabilities of FE discretization by applying finer mesh where sharp phenomena takes place (solid-liquid interfaces, boundary layers, recirculation zones, ...) and coarser mesh elsewhere (in the solid, in the bulk of the fluid region where the gradient are lower than the boundary layer region, ...) is helpful to reduce the degree of freedom involved in the numerical resolution and thus reduce the computational time.
We use a finite-element method that was implemented using the open-source software FreeFem++ \citep{freefem,hecht-2012-JNM}, using a large variety of triangular finite elements  to solve partial differential equations. 

FreeFem++  is an integrated product with its own high level programming language and a syntax close to mathematical formulations, making the implementation of numerical algorithms very easy. Among the numerous numerical tools offered by FreeFem++, the use of the powerful mesh adaptivity function proved mandatory in this study to obtain accurate results within reasonable computational time.
The numerical code was optimized to afford the mesh refinement every time step:  the mesh density was increased around  the phase change interfaces,  offering an optimal resolution of the large gradients of all regularized functions ($S, K, C, \varphi$), while  the mesh was de-refined (larger triangles) in the solid part, where a coarser mesh could be used. A simulation using a globally refined mesh would require a prohibitive computational time for an equivalent accuracy of the melting front resolution. Similar algorithms based on FreeFem++  were successfully used for solving different systems of equations with locally sharp variation of the solution, such as Gross-Pitaevskii equation \citep{dan-2010-JCP,dan-2016-CPC} or  Laplace equations with nonlinear source terms \citep{dan-2013-AMM}. 

The space discretization is based on Taylor-Hood finite elements, approximating  the velocity with $P_{2}$ Lagrange finite elements (piecewise quadratic), and the
the pressure with the $P_{1}$ finite elements (piecewise linear). The temperature and the enthalpy are discretized using $P_2$ finite elements.  
This discretization is second order accurate in space.
We also use a second order accurate discretization in time.
\cite{aldbaissy2018full} have compared analytically and numerically the first and second order discretization of the time-dependent Boussinesq problem, and have concluded that the second order scheme is much better than the first order in terms of CPU time with the same precision.
A fully implicit backward second order scheme (BDF2 or GEAR) is employed in the present study. The time derivative of a variable $\phi$ is approximated  by:
\begin{equation}	
\label{eq-Gear}
	\frac{d\phi}{dt} \simeq \frac{3\phi^{n+1} - 4\phi^{n}+ \phi^{n-1}}{2\delta t},
\end{equation}
computing the solution $\phi^{n+1}$ at time  $t_{n+1}=(n+1) \delta t$ by using two previous states ($\phi^{n}, \phi^{n-1}$). We use this scheme to advance in time both velocity ($\phi=\vec{u}$) and temperature fields  ($\phi=\theta$).  The other terms in eqs. (\ref{eq-qmvt})-(\ref{eq-energ}) are treated implicitly (\ie taken at time $t_{n+1}$). The resulting non-linear equations are solved using a Newton algorithm. 
Some authors used a Richardson extrapolation \citep{Belhamadia2012,wang2010numerical} in the momentum equation by extrapolating U from previous time steps.
We have tested this approach but the results exhibit less accurate solutions and the computations requests small $\delta t$.
In addition, a projection algorithm with an explicit discretization of the Navier-Stokes-Boussinesq equations was also tested, using the Adams-Bashforth and Crank-Nicolson second order scheme.
The main drawback is the very small $\delta t$ ($\sim 10^{-6}$ vs $10^{-1}$ for the fully implicit discretization) needed to ensure convergence at each time steps.
A last alternative for the treatment of non-linear terms in the momentum equation is offered by the characteristics Galerkin method \cite{Pironneau92}.
This method will be discussed in detail in sec. \ref{sec-charac-FreeFem}.

Finally, a supplementary difficulty comes from the one-domain method which need techniques to bring the velocity to zero in the solid region.
The switch-off technique, the variable viscosity approach and the Carman-Kozeny penalty term are the most used in the literature.
The physical meaning of the variable viscosity formulation and the Carman-Kozeny penalty term is fundamentally different.
The Carman-Kozeny approach considers the mushy zone as a porous media, i.e the solid is stationary and the liquid flows through the porous structures, 
while the variable viscosity formulation treats the mushy zone as a mixture of solid crystals and liquid, permitting thus a movement of both the solid and the liquid.
The $\mu$-based method was investigated by \cite{dan-2014-JCP} and the Carman-Kozeny penalty method is investigated in the present work.
We note however that techniques in FV methods based on the modification of the numerical algorithm by cutting-off the velocity in the solid by a relaxation scheme also exists.

\newpage
\section{Finite element algorithm} \label{sec-FE-algo}

To solve the system of equations (\ref{eq-qmvt})-(\ref{eq-energ}) we use a finite-element method that was implemented using the open-source software \ff \citep{freefem,hecht-2012-JNM}.
Finite-element methods for solving Navier-Stokes type systems of equations  are generally based on a separate discretization of the temporal derivative (using finite difference, splitting or characteristics methods) and the generalization of the Stokes problem for the resulting system \citep{Temam,GRaviart,Quarteroni}. 
We use the second-order implicit finite-difference discretization (\ref{eq-Gear}) of the temporal derivative and obtain the time semi-discretization of the single-domain model (\ref{eq-qmvt})-(\ref{eq-energ}):
\begin{eqnarray} \label{eq-time-disc1}
\nabla\cdot \vec{u}^{n+1} + {\gamma} p^{n+1} &=& 0, \\ %\nonumber
\frac{3}{2} \frac{\vec{u}^{n+1}}{\delta t} +(\vec{u}^{n+1}\cdot\nabla) \vec{u}^{n+1} +\nabla p^{n+1} 
- {\frac{1}{Re} \nabla^2 \vec u^{n+1}}  & & \\ \nonumber 
- A(\theta^{n+1})\vec u^{n+1}- f_B(\theta^{n+1}) \, \vec{e}_y &=&  \\ \nonumber
2 \frac{\vec{u}^{n}}{\delta t}-\frac{\vec{u}^{n-1}}{2\delta t},\\ \label{eq-time-disc3}
\frac{3}{2} \frac{\theta^{n+1} + S(\theta^{n+1})}{\delta t} +
\nabla\cdot\left(\vec{u}^{n+1} \theta^{n+1}\right)
- \nabla \cdot\left( \frac{K}{RePr} \nabla \theta^{n+1} \right) &=& \\  \nonumber
2\frac{ \theta^{n} + S(\theta^{n})}{\delta t}-\frac{ \theta^{n-1} + S(\theta^{n-1}) }{2\delta t}.
\end{eqnarray}
The penalty parameter $\gamma$ takes very low values ($\gamma=10^{-7}$)  to ensure a pressure field with zero average and, at the algebraic level, fulfill the diagonal of the pressure term.  
This system of non-linear equations is solved at time  $t_{n+1}=(n+1) \delta t$, using two  previous states at $t_{n}$ and $t_{n-1}$.

The space discretization of variables over the domain $\Omega=[0,1]^2$ uses a finite-element method based on a weak formulation of the system of eqs. (\ref{eq-time-disc1})-(\ref{eq-time-disc3}). 
We consider homogeneous Dirichlet boundary conditions for the velocity, \ie $\vec{u}=0$ on $\pl \Omega$, and set the classical Hilbert spaces for the velocity and pressure:
\begin{equation}
\vec{V}=V\times V, \, V=H^1_0(\Omega), \quad Q=\left\{q\in L^2(\Omega)\left|\; \int_{\Omega}q=0\right.\right\}
\end{equation}
Following the generalization of the Stokes problem \citep{Temam,GRaviart,Quarteroni}, the variational formulation of the system  (\ref{eq-time-disc1})-(\ref{eq-time-disc3}) can be written as: find $(\vec{u}^{n+1}, p^{n+1}, \theta^{n+1}) \in \vec{V}\times Q\times V$, such that:
\begin{eqnarray}
\label{eq-weak-all}
b\left(\vec{u}^{n+1}, q\right) - \gamma (p^{n+1},q)&=& \\ \nonumber
0, \, \forall \, q \in Q \\ %\nonumber
\frac{3}{2 \delta t} \left(\vec{u}^{n+1},\vec{v}\right) + c\left(\vec{u}^{n+1} ; \vec{u}^{n+1}, \vec{v} \right) +
{a\left(\vec{u}^{n+1}, \vec{v}\right)} & &\\ \nonumber
- (A(\theta^{n+1}) \, \vec u^{n+1},\vec v)+ b\left(\vec{v}, p^{n+1}\right)
- {\left(f_B(\theta^{n+1}) \, \vec{e}_y,\vec{v}\right)}
&=& \\ \nonumber
\frac{2}{\delta t} \left(\vec{u}^{n},\vec{v}\right) 
- \frac{1}{2 \delta t} \left(\vec{u}^{n-1},\vec{v}\right), \, \forall \, \vec{v} \in \vec{V}\\ \label{eq-weak-energy}  %\nonumber
\frac{3}{2 \delta t} \left(\theta^{n+1} + S(\theta^{n+1}), \phi\right)
+\left(\vec{u}^{n+1} \cdot \nabla \theta^{n+1} , \phi
\right) +
\left( \frac{K}{Re Pr} \nabla \theta^{n+1}, \nabla \phi \right) &=& \\  \nonumber
\frac{2}{\delta t} \left( \theta^{n}+S(\theta^n), \phi\right)
- \frac{1}{2 \delta t} \left( \theta^{n-1}+S(\theta^{n-1}), \phi\right),\, \forall \, \phi \in V,
\end{eqnarray}
where {$(u , v)=\int_{\Omega} u\cdot v$} denotes the scalar product in $L^2(\Omega)$ or $\left(L^2(\Omega)\right)^2$; the bilinear forms $a, b$ and trilinear form $c$ are defined as \cite{GRaviart,Quarteroni}:
%\begingroup \small{
%	\begin{eqnarray*}\nonumber
%		a: \vec{V} \times \vec{V} \rightarrow \R, & & {a(\vec{u},\vec{v})= \int_{\Omega}  
%			 \vec{D}(\vec{u}) : \vec{D}(\vec{v}) = \int_{\Omega}   \sum_{i,j=1}^2{D}_{ij}(\vec{u}) {D}_{ij}(\vec{v})},\\ 
%		b: \vec{V} \times Q \rightarrow \R, & & b(\vec{u},q) = -\int_{\Omega}\nabla \cdot\vec{u}\, q =
%		-\sum_{i=1}^2\int_{\Omega}\pl_i u_i\cdot q, \\ \nonumber
%		c: \vec{V} \times \vec{V} \times \vec{V} \rightarrow \R, & & c(\vec{w}; \vec{z}, \vec{v})=\int_{\Omega} \left[\left(\vec{w} \cdot \nabla\right) \vec{z}\right] \cdot\vec{v}
%		=\sum_{i,j=1}^2\int_{\Omega} w_j (\pl_j z_i) v_i,
%		\label{eq-biforms}
%	\end{eqnarray*}
%}
\begingroup{ \small
	\begin{eqnarray*}\nonumber
		a: \vec{V} \times \vec{V} \rightarrow \R, & & a(\vec{u},\vec{v})= \int_{\Omega}  
			 \vec{\nabla^t} \vec{u} : \vec{\nabla} \vec{v} = \sum_{i,j=1}^2 \int_{\Omega}   \pl_j u_j \cdot \pl_j v_i,\\ 
		b: \vec{V} \times Q \rightarrow \R, & & b(\vec{u},q) = -\int_{\Omega}\nabla \cdot\vec{u}\, q =
		-\sum_{i=1}^2\int_{\Omega}\pl_i u_i\cdot q, \\ \nonumber
		c: \vec{V} \times \vec{V} \times \vec{V} \rightarrow \R, & & c(\vec{w}; \vec{z}, \vec{v})=\int_{\Omega} \left[\left(\vec{w} \cdot \nabla\right) \vec{z}\right] \cdot\vec{v}
		=\sum_{i,j=1}^2\int_{\Omega} w_j (\pl_j z_i) v_i.
		\label{eq-biforms}
	\end{eqnarray*}
}
%$\vec D(\vec u) = (1/2)\left( \nabla \vec u + \nabla \vec u^T \right)$ is the rate of strain symmetric tensor with components
%$D_{ij}(\vec u) = (1/2) \left( \partial u_i/ \partial x_j + \partial u_j/ \partial x_i \right) $.

The system of non-linear eqs. (\ref{eq-weak-all}) - (\ref{eq-weak-energy}) is solved using a Newton method. To advance the solution from time $t_n$ to $t_{n+1}$, we start from an initial guess $w_0 = (\vec{u}^{n}, p^{n}, \theta^{n})$ (which is the solution at $t_n$), and construct the Newton sequence $w_k = (u_k, p_k, \theta_k)$ by solving for each inner iteration $k$:
\begin{equation}
D_w{\cal F} (w_k ) w_{k+1} = D_w{\cal F}(w_k) w_k -  {\cal F}(w_k).
\label{eq-newton-w}
\end{equation}
$D_w{\cal F}$ is the linear operator representing the differential of $\cal F$.
Eq. (\ref{eq-newton-w}) can be rewritten as follow when applied to the discretized equation, in which eqs. (\ref{eq-weak-all}) - (\ref{eq-weak-energy}) are regarded as ${\cal F}(w) = 0$:
%\begingroup \small{
%	\begin{eqnarray} \label{eq-newton-C1}
%	b\left(\vec{u}_{k+1}, q\right) - \gamma (p_{k+1},q) &=& 0, \\ %\nonumber
%	\frac{3}{2 \delta t} \left(\vec{u}_{k+1},\vec{v}\right)
%	+ c\left(\vec{u}_{k+1} ; \vec{u}_{k}, \vec{v} \right)
%	&+& c\left(\vec{u}_{k} ; \vec{u}_{k+1}, \vec{v} \right)\\ \nonumber
%	+ 
%	a\left( \vec{u}_{k+1}, \vec{v}\right)
%	- \left(\frac{d A}{d\theta}(\theta_k)\, \theta_{k+1} \, \vec{u}_k, \vec{v}\right)
%	&-& \left(A(\theta_k) \, \vec{u}_{k+1}, \vec{v}\right) + b\left(\vec{v}, p_{k+1}\right)  \\ \nonumber
%	- {\left(\frac{df_B}{d\theta}(\theta_k)\, \theta_{k+1} \, \vec{e}_y, \vec{v}\right)} &=&  
%	\frac{1}{\delta t} \left( 2 \vec{u}^n - \frac{1}{2} \vec{u}^{n-1},\vec{v}\right) \\ \nonumber
%	+ c\left(\vec{u}_k ; \vec{u}_{k}, \vec{v} \right) 
%	&-& \left(\frac{d A}{d\theta}(\theta_k)\, \theta_{k} \, \vec{u}_k, \vec{v}\right), \,\,\, \\  %\nonumber
%	\frac{3}{2\delta t} \left(\theta_{k+1} + \frac{dS}{d\theta}(\theta_k)\, \theta_{k+1}, \phi\right)
%	+\left(\vec{u}_{k} \cdot \nabla \theta_{k+1} , \phi  \right)
%	&+& \left( \vec{u}_{k+1} \cdot \nabla  \theta_k , \phi \right) \\ \nonumber
%	+ \left( \frac{K}{Re Pr} \nabla \theta_{k+1}, \nabla \phi \right) 
%	&=&
%	\frac{2}{\delta t} \left(\theta^n + S(\theta^n) , \phi\right)  \\ \nonumber
%	+\left(\vec{u}_{k} \cdot \nabla  \theta_k, \phi \right)
%	+ \frac{3}{2 \delta t} \left(\frac{dS}{d\theta}(\theta_k)\, \theta_{k}  - S(\theta^n) ,\, \phi\right) 
%	&-& \frac{1}{2 \delta t} \left(\theta^{n-1} + S(\theta^{n-1}),\, \phi\right).\,\,\,
%	\end{eqnarray}
%} \endgroup
\begingroup \small{
	\begin{eqnarray} \label{eq-newton-C1}
	b\left(\vec{u}_{k+1}, q\right) - \gamma (p_{k+1},q) &=& 0, \\ %\nonumber
	\frac{3}{2 \delta t} \left(\vec{u}_{k+1},\vec{v}\right)
	+ c\left(\vec{u}_{k+1} ; \vec{u}_{k}, \vec{v} \right)
	&+& c\left(\vec{u}_{k} ; \vec{u}_{k+1}, \vec{v} \right)\\ \nonumber
	+ 
	\frac{1}{Re} a\left( \vec{u}_{k+1}, \vec{v}\right)
	- \left(\frac{d A}{d\theta}(\theta_k)\, \theta_{k+1} \, \vec{u}_k, \vec{v}\right)
	&-& \left(A(\theta_k) \, \vec{u}_{k+1}, \vec{v}\right) + b\left(\vec{v}, p_{k+1}\right)  \\  \nonumber
	- {\left(\frac{df_B}{d\theta}(\theta_k)\, \theta_{k+1} \, \vec{e}_y, \vec{v}\right)} &=&  
	\frac{1}{\delta t} \left( 2 \vec{u}^n - \frac{1}{2} \vec{u}^{n-1},\vec{v}\right) \\ \label{eq-newton-C2} \nonumber
	+ c\left(\vec{u}_k ; \vec{u}_{k}, \vec{v} \right) - \left(\frac{d A}{d\theta}(\theta_k)\, \theta_{k} \, \vec{u}_k, \vec{v}\right)
	&-& \left( \left(\frac{df_B}{d\theta}(\theta_k)\, \theta_{k} - f_B(\theta_k)\right)\, \vec{e}_y,\vec{v}\right), \,\,\, \\   %\nonumber
	\frac{3}{2\delta t} \left(\theta_{k+1} + \frac{dS}{d\theta}(\theta_k)\, \theta_{k+1}, \phi\right)
	+\left(\vec{u}_{k} \cdot \nabla \theta_{k+1} , \phi  \right)
	&+& \left( \vec{u}_{k+1} \cdot \nabla  \theta_k , \phi \right) \\ \nonumber
	+ \left( \frac{K}{Re Pr} \nabla \theta_{k+1}, \nabla \phi \right) 
	&=&
	\frac{2}{\delta t} \left(\theta^n + S(\theta^n) , \phi\right)  \\ \label{eq-newton-C3} \nonumber
	+\left(\vec{u}_{k} \cdot \nabla  \theta_k, \phi \right)
	+ \frac{3}{2 \delta t} \left(\frac{dS}{d\theta}(\theta_k)\, \theta_{k}  - S(\theta_k),\, \phi\right) 
	&-& \frac{1}{2 \delta t} \left(\theta^{n-1} + S(\theta^{n-1}),\, \phi\right).\,\,\,
	\end{eqnarray}
} \endgroup
We underline the fact that the Newton loop (following $k$) has to be iterated until convergence for each time step $\delta t$ following the algorithm:
Note that the last term of Eq. (\ref{eq-newton-C2}) cancels in the case of a linear Boussinesq force $f_B$  (see Eq. (\ref{eq-RePr})); this is not the case when non-linear variations of the density of the liquid are considered (convection or solidification of water).  Note also that the previous system of equations (\ref{eq-newton-C1})-(\ref{eq-newton-C3}) depends only on $\vec{u}^n$, $\vec{u}^{n-1}$, $\theta^n$ and $\theta^{n-1}$ and is independent of $p^n$, the pressure being in this approach a Lagrange multiplier for the divergence free constraint. 

The Newton loop (following $k$) has to be iterated until convergence for each time step $\delta t$ following the algorithm:
\begin{equation} \label{eq-Newton-algo}
\begin{tabular}{||ll}
\multicolumn{2}{||l}{Navier-Stokes time loop following $n$}\\
\multicolumn{2}{||l}{set  $w_0=(\vec{u}^{n}, p^{n}, \theta^{n})$}\\
& \begin{tabular}[t]{||ll}
\multicolumn{2}{||l}{Newton iterations  following $k$}\\
&solve (\ref{eq-newton-C1}) to get $ \vec w_{k+1} $\\
\multicolumn{2}{||l}{stop when  $\| \vec w_{k+1} - \vec w_k \| < \xi_N$}
\end{tabular}\\
\multicolumn{2}{||l}{actualize $(\vec{u}^{n+1}, p^{n+1}, \theta^{n+1}) = w_{k+1}$}
\end{tabular}
\end{equation}
%It is interesting to note that the previous system of equations depends only on $\vec{u}^n$, $\vec{u}^{n-1}$, $\theta^n$ and $\theta^{n-1}$ and is independent of $p^n$, the pressure being in this approach a Lagrange multiplier for the divergence free constraint. 
The \ff syntax to implement the Newton algorithm is very close to the mathematical formulation given above. 
	After defining a vectorial finite-element space \texttt{fespace Wh(Th,[P2,P2,P1,P1]);}, associated to the mesh \texttt{Th}, we define the velocity, pressure and temperature variables in a compact manner by \texttt{Wh [u1,u2,p,T];}. Corresponding test functions are defined similarly. It is then very easy to define a \texttt{problem} formulation in \ff and include all the terms of the algorithm  (\ref{eq-newton-C1})-(\ref{eq-newton-C3}). This makes the lecture of the programs very intuitive by comparing each term to its mathematical expression. New terms could be added to the variational formulation expressed in the \texttt{problem} structure, without affecting other parts of the program. Consequently, the implementation of new models  or numerical methods for this problem is greatly facilitated by this modular structure of programs.

\section{Characteristics method for the convective terms} \label{sec-charac-FreeFem}

An alternative for the treatment of non-linear terms in the momentum equation is offered by the characteristics Galerkin method \cite{Pironneau92}. Defining the characteristic flow (passing at time $t$ through the point $\vec{x}$)
\begin{equation}
\left\{
\begin{array}{l}\vspace{0.2cm}
\ds\frac{\partial \vec{X} }{\partial \tau}(\tau, t, \vec{x})=\vec{u} (\tau,\vec{X} (\tau,t, \vec{x})), \quad \tau\in (0,t_{max})\\
\vec{X} (t, t, \vec{x}) = \vec{x},
\end{array}
\right.
\label{eq-charactD}
\end{equation}
one can express the substantial (total) derivative of any function $\Phi(t,\vec{x})$ as
\begin{equation}
\frac{D\Phi}{Dt}(t,\vec{x})=\left( \frac{\partial \Phi}{\partial
	t}+\vec{u}\cdot\nabla \Phi \right)(t,\vec{x})=\frac{\partial}{\partial
	t}\left(\Phi(\tau, \vec{X}(\tau,t, \vec{x}))\right)|_{\tau=t},
\end{equation}
and finally use the time discretization:
\begin{equation}
\left(\frac{D\Phi}{Dt}\right)^{n+1}(\vec{x})\approx\frac{\Phi^{n+1}(\vec{x})-\Phi^{n}\circ \vec{X}^n(\vec{x})}{\delta t},
\end{equation}
with $\vec{X}^n(\vec{x})$ a suitable approximation of $\vec{X}(t_n,t_{n+1},\vec{x})$, obtained by an integration back in time of (\ref{eq-charact}) from $t_{n+1}$ to $t_n$ for each grid point $\vec{x}$. The Galerkin characteristic method is implemented in Freefem++ as an operator computing $\Phi \circ \vec{X}^n$ for given mesh, convection velocity field and time step.

The weak formulation  (\ref{eq-weak-all}) becomes after using the characteristics method:
\begin{eqnarray}
\label{eq-charact}
b\left(\vec{u}^{n+1}, q\right) - \gamma (p^{n+1},q)&=& \\ \nonumber
0, \, \forall \, q \in Q \\ \nonumber
\frac{3}{2 \delta t} \left(\vec{u}^{n+1},\vec{v}\right) +
a\left(\vec{u}^{n+1}, \vec{v}\right) 
+ b\left(\vec{v}, p^{n+1}\right) 
- f_B(\theta^{n+1}) \left(\vec{e}_y,\vec{v}\right)
&=& \\ \nonumber
\frac{2}{\delta t} \left(\vec{u}^{n}\circ \vec{X}^n,\vec{v}\right) - \frac{1}{2 \delta t} \left(\vec{u}^{n-1}\circ \vec{X}^{n-1},\vec{v}\right), \, \forall \, \vec{v} \in \vec{V}\\  \nonumber
\frac{3}{2 \delta t} \left(\theta^{n+1}, \phi\right)  
+
\left( \frac{K}{\Prd} \nabla \theta^{n+1}, \nabla \phi \right) &=& \\ \nonumber
 \frac{2}{\delta t} \left(\theta^{n}\circ \vec{X}^n, \phi\right) -  \frac{1}{2 \delta t} \left(\theta^{n-1}\circ \vec{X}^{n-1}, \phi\right), \, \forall \, \phi \in V,
\end{eqnarray}
The system (\ref{eq-charact}) is then solved to advance the solution from $t_n$ to $t_{n+1}$ in one step. The drawback of the method is that it requires small time steps for accurately computing of the convection terms. 

 \section{Mesh adaptivity} \label{subs:FEadapt}
We use the standard mesh adaptivity function (\texttt{adaptmesh}) offered by \ff \citep{hecht-2012-JNM}. The key idea implemented in this function (see also \cite{hecht-1996-missi,hecht-2000-ijnmf,hecht-1997-aiaa,george-1998,frey-george-1999,moham-piron-2000})  is to modify the scalar product used in the automatic mesh generator to evaluate distance and volume.  Equilateral elements are thus constructed, accordingly to the new metric.  The scalar
product is based on the evaluation of the Hessian $\mathcal{H}$ of the variables of the problem. For example, for a P$_1$ discretization of a
variable $\chi$, the interpolation error is bounded by:
\begin{equation}
{\cal E } = |\chi - \Pi_h \chi |_0 \leq c \sup_{T\in \mathcal{T}_{h}} \sup_{x,y,z\in T}   |\mathcal{H}(x)|(y-z , y-z),
\label{eq1}
\end{equation}
where $\Pi_h \chi $ is the $P_1$ interpolate  of $\chi$, $ |\mathcal{H}(x)|$ is the  Hessian of $\chi$ at point $x$ after being made positive definite.
Using the Delaunay algorithm (\eg \cite{george-1998}) to generate a  trianguler mesh with edges close to the unit length  in the metric  $\mathcal{M} ={|\mathcal{H}| \over (c {\cal{E}})}$ will result in a equally distributed  interpolation error ${\cal E}$ over the edges $a_i$ of the mesh. More precisely, we get
\begin{equation}
{1 \over c {\cal E}} a_i^T {\cal M } a_i \le 1.
\end{equation}


The previous approach could be generalized for a vector variable $\chi=[\chi_1, \chi_2]$. After computing the metrics $\mathcal{M}_{1}$ and $\mathcal{M}_{2}$ for each variable, we define a metric intersection  $\mathcal{M} = \mathcal{M}_{1} \cap \mathcal{M}_{2}$,
such that the unit ball of $\mathcal{M}$ is included in  the intersection of the two  unit balls  of metrics $\mathcal{M}_{2}$ and $\mathcal{M}_{1}$
(for details, see the  procedure defined in \cite{frey-george-1999}).

%For this purpose, we use the  procedure defined in \cite{frey-george-1999}.
%Let $\lambda _{i}^{j}$ and $v_{i}^{j}$, ($i,j=1,2$) be the eigenvalues and
%eigenvectors of ${\cal M}_{j}$, $j=1,2$. The intersection metric ($\hat{{\cal M}}$) is defined by
%\begin{equation}
%\hat{{\cal M}}={\frac{{\hat{{\cal M}_{1}}+\hat{{\cal M}_{2}}}}{2}}.
%\end{equation}
%where $\hat{{\cal M}_{1}}$ (resp. $\hat{{\cal M}_{2}}$) has the same
%eigenvectors than ${\cal M}_{1}$, (resp. ${\cal M}_{2}$ ) but with
%eigenvalues defined by:
%\begin{equation}
%\tilde{\lambda _{i}^{1}}=\max (\lambda _{i}^{1},{v_{i}^{1}}^{T}{\cal M}_{2}v_{i}^{1}),\quad i=1,2.
%\end{equation}

The \texttt{adaptmesh} function offers the possibility to take into account several metrics computed from different variables monitoring the evolution of the phase-change system. For natural convection system, the mesh will be adapted using the values of the two velocity components and the temperature. For phase-change systems, to accurately track the solid-liquid interface we add the variation of the enthalphy source term in the adaptivity criterion. For  water systems (convection or freezing), we also add an extra function tracking the anomalous change of density around $4 {^o}C$. To reduce the impact of the interpolation on the global accuracy for time-depending problems, we consider, for each variable used for adaptivity,  the metrics computed at actual ($t_{n+1}$) and  previous ($t_{n}$) time instants (see also \cite{Belhamadia2004_S}). The anisotropy of the mesh is a parameter of the algorithm and it was set to values close to 1. This is an inevitable limitation since we also impose the minimum edge-length of triangles to avoid too large meshes.
The capabilities of the mesh adaptivity algorithm are illustrated in Chap. \ref{chap-MELTING}.

%Mesh adaptivity by metric control is a standard function offered by FreeFem++ \citep{hecht-2012-JNM}. The key idea for the mesh adaptivity (see also \cite{hecht-2000-ijnmf,hecht-1997-aiaa,george-1998}) is to modify the scalar product used in an automatic mesh generator to evaluate distance and volume, in order to  construct equilateral elements according to a new adequate metric.  The scalar
%product is based on the evaluation of the Hessian $\mathcal{H}$ of the variables of the problem. Indeed, for a $P_1$ discretization of a
%variable $\chi$, the interpolation error is bounded by:
%\begin{equation}
%{\cal E } = |\chi - \Pi_h \chi |_0 \leq c \sup_{T\in \mathcal{T}_{h}} \sup_{x,y,z\in T}   |\mathcal{H}(x)|(y-z , y-z)
%\label{eq1}
%\end{equation}
%where $\Pi_h \chi $ is the $P_1$ interpolate  of $\chi$, $ |\mathcal{H}(x)|$ is the  Hessian of $\chi$ at point $x$ after being made positive definite.
%We can infer that, if we generate, using  a Delaunay procedure (e.g. \cite{george-1998}), a  mesh with edges close to the unit length  in the metric  $\mathcal{M} ={|\mathcal{H}| \over (c {\cal{E}})}$, the interpolation error ${\cal E}$ will be equally distributed over the edges $a_i$ of the mesh. More precisely, we have
%\begin{equation}
%{1 \over c {\cal E}} a_i^T {\cal M } a_i \le 1.
%\end{equation}
%
%The previous approach could be generalized for a vector variable $\chi=[\chi_1, \chi_2]$.
%After computing the metrics $\mathcal{M}_{1}$ and $\mathcal{M}_{2}$ for each variable, we define a metric intersection  $\mathcal{M} = \mathcal{M}_{1} \cap \mathcal{M}_{2}$,
%such that the unit ball of $\mathcal{M}$ is included in  the intersection of the two  unit balls  of metrics $\mathcal{M}_{2}$ and $\mathcal{M}_{1}$ (for details, see the  procedure defined in \cite{frey-george-1999}). 

%%%%%%%%%%%%%%%%%%%%%%%%%%%%%%%%%%%%%%%%%%%%%%%%%%%%%%%%%%%
%\newpage
%\section{Numerical resolution for large scale simulation}

\section{Domain decomposition method with FreeFem++: FFDDM}
Solving the Navier-Stokes-Boussinesq equation in three-dimensional configurations could generate a large problem size.
The natural convection of air in a cube of dimensions $[0,1]^3$ with $40 \times 40 \times 40$ uniform grids involve $3$ millions of unknowns in the linear system when $P_1$ finite element is considered for the temperature.
For such a large size of problem, memory lack issue can rapidly arise with sequential algorithms.
It is thus essential to distribute data among several processors.
A natural approach is the domain decomposition method.

FFDDM (FreeFem++ Domain Decomposition Method) is a parallel part of FreeFem++ allowing to use parallel solver in FreeFem++.
The data distribution among the processor is done via an overlapping domain decomposition and a related linear algebra.
The linear system is then solved by using the domain decomposition method as preconditioners to the GMRES Krylov method.

We use in our simulations the Optimized Restricted Additive Schwarz (ORAS) preconditionner.
To solve the linear equation $A x = rhs$, the ORAS preconditionner reads:
\begin{equation}
   M_{RAS}^{-1} = \sum_{j=1}^{\mathcal{N}} R^T_j D_j (R_j A R^T_j)^{-1} R_j,
\end{equation}
$R_j$ denote the restriction operators and $D_j$ are square diagonal matrices.
Local matrices are defined as:
\begin{equation}
   A_j = R_i A R_i^T.
\end{equation}
The duplicated unknowns due to the overlap between subdomains are coupled via a partition of unity:
\begin{equation}
   I = \sum_{i=1}^{\mathcal{N}} R_i^T D_i R_i
\end{equation}
Thus, the global solutions $U$ is defined as:
\begin{equation}
  U = \sum_{i=1}^{\mathcal{N}} R_i^T D_i R_i U =  \sum_{i=1}^{\mathcal{N}} R_i^T D_i U_i
\end{equation}

