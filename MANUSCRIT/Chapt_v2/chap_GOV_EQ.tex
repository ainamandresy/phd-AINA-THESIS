%%%%%%%%%%%%%%%%%%don't forget if needed %%%%%%%%%%%%%%%%%%%%%
%\section[toc version]{title version%
%              \sectionmark{head version}}
%\sectionmark{head version}
%%%%%%%%%%%%%%%%%%%%%%%%%%%%%%%%%%%%%%%%%%%%%%%%%%%%%%%%%%%%%%
\def\titcourt{Numerical resolution of the Navier-Stokes-Boussinesq model}
\def\titlong{Numerical resolution of the Navier-Stokes-Boussinesq model}
%%%%%%%%%%%%%%%%%%%%%%%%%%%%%%%%%%%%%%%%%%%%%%%%%%%%%%%%%%%%%%%%
\chapter[\titlong]{\titlong%
              \chaptermark{\titcourt}}
\chaptermark{\titcourt}
\label{chap-NSB}
%%%%%%%%%%%%%%%%%%%%%%%%%%%%%%%%%%%%%%%%%%%%%%%%%%%%%%%%%%%%%%%%
%%%%%%%%%%%%%%%%%%%%%%%%%%%%%%%%%%%%%%%%%%%%%%%%%%%%%%%%%%%%%%%%

%\section{Governing equations} \label{sec-gov-eq}
%%%%%%%%%%%%%%%%%%%%%%%%%%%%

We consider a solid-liquid system placed in a two-dimensional square cavity of height $H$. In the following, subscripts $s$ and $l$ will refer to the solid and liquid phases, respectively. 

For the numerical implementation, it is convenient to adopt a single-domain approach to describe both phases using the same system of equations. 
The model is based on the Navier-Stokes equations with Boussinesq approximation, which is the natural description of the fluid flow with natural convection. 
A penalty term is added to the momentum equations to bring the velocity to zero inside the solid region. 
For the energy conservation equation, an enthalpy method is used to model the phase change process. The single-domain model is described in detail in the following sections.

\section{Enthalpy method}

The phase change process is modelled using an enthalpy method \citep{voller1987pcm,Cao1989,Cao1990} with temperature-based formulation. We start from the energy equation:
\begin{equation}
\label{eq-energie}
   \frac{\partial (\rho h)}{\partial t_{\varphi}} + \nabla \cdot(\rho h \vec{\tilde{u}}) - \nabla \cdot (k \nabla T) = 0,
\end{equation}
where $t_{\varphi}$ is the physical time, $h$ the enthalpy, $\rho$ the density, $\vec{\tilde{u}}$  the velocity vector, $T$ the temperature and $k$ the thermal conductivity. 
To make Equation (\ref{eq-energie})  valid for the entire domain containing both liquid and solid phases, the total enthalpy $h$ is regarded as the sum of the sensible heat and the latent heat:
\begin{equation}
\label{eq-enth-model}
  h = c ( T + s(T) ),
\end{equation} 
with $c$ the local specific heat. The function $s(T)$ is introduced to model the jump of the enthalpy due to the phase change and is theoretically a Heaviside step function depending on the temperature: it takes the zero value in the solid region and a large value in the liquid, equal to $h_{sl}/c$, with $h_{sl}$ the latent heat of fusion. 
Linear  \citep{voller1987pcm,Wang2010} or smoother functions \citep{dan-2014-JCP} can be used to regularize $s(T)$ and also the jump of material properties (from solid to liquid). 
We use a regularization of all step-type functions by a continuous and differentiable hyperbolic-tangent function suggested by \cite{dan-2014-JCP} (see below). 
\Blue{%Moreover, we assume that the undercooling problem is here negligible since only pure and homogeneous materials are considered.
We assume moreover that the undercooling phenomenon is negligible (see also \cite{wang2010numerical,kowalewski2004phase}).}

Equation (\ref{eq-energie}) can be further simplified by considering the following assumptions: (i) the density difference between solid and liquid phases is negligible, \ie $\rho_l=\rho_s=\rho$ is constant; (ii) the regularization zone is narrow and the velocity inside this zone is very low. 
Consequently, the final expression of the energy equation is obtained by combining (\ref{eq-enth-model})  and (\ref{eq-energie}) and  neglecting the convection term $\nabla \cdot ( c s \vec{\tilde{u}})$\footnote{In the liquid phase, $\nabla \cdot ( c s \vec{\tilde{u}})  = h_{sl} \nabla \cdot  \vec{\tilde{u}}=0$; in the solid phase, $s=0$; in the regularization region, it is assumed that $\vec{\tilde{u}}=0.$}:
\begin{equation}\label{eq-energie-enth-model}
\frac{\partial \left(c T\right)}{\partial t_{\varphi}} + \nabla \cdot\left( c T \vec{\tilde{u}}\right) -
\nabla \cdot\left( \frac{k}{\rho} \nabla T \right) +  \frac{\partial \left(c s\right)}{\partial t_{\varphi}}  = 0.
\end{equation}
The essential feature of the current approach is that the phase change front is not tracked explicitly but is instead recovered a posteriori from the computed temperature field.
The phase-change occurs over a temperature interval $  T \in [T_f - T_{\varepsilon1}, T_f + T_{\varepsilon2}] $ around the temperature of fusion $T_f$.
For non-isothermal phase-change PCM, $T_{\varepsilon1}$ and $T_{\varepsilon2}$ correspond to the solidus and the liquidus temperature of the material.
However, for a pure material involving a unique phase-change temperature, $T_{\varepsilon1}$ and $T_{\varepsilon2}$ represent an artificial mushy zone used to regularize discontinuous parameters and should be set as small as possible.

\section{Navier-Stokes equations with Boussinesq approximation}
 

The natural convection in the liquid part of the system is modelled using the incompressible Navier-Stokes equations, with  Boussinesq approximation for buoyancy effects. To make this model valid for both liquid and solid phases, the momentum equation is modified as follows:
\begin{equation}\label{eq-momentum-conserv-1}
  \rho \left( \frac{\partial \vec{\tilde{u}}}{\partial t_{\varphi}} +   {(\vec{\tilde{u}}\cdot\nabla ) \vec{\tilde{u}}} \right) + \nabla P - \mu_{l}  {\nabla^2 \vec{\tilde{u}}} 
- \rho g \vec{e}_y= A(T) \vec{\tilde{u}},
\end{equation}
where $P$ denotes the pressure, $\mu_{l}$ the dynamic viscosity of the liquid (assumed to be constant).  
%and $f_B(T)$ the Boussinesq force. 

The penalty term $A(T) \vec{\tilde{u}}$ is artificially introduced in (\ref{eq-momentum-conserv-1}) to extend this equation in the solid phase, where the velocity, pressure, viscosity and Boussinesq force are meaningless.  Consequently, $A(T)$  is modelled to vanish in the liquid, where the Navier-Stokes-Boussinesq momentum equation is recovered. A large value of $A(T)$ is imposed in the solid, reducing the momentum eq. (\ref{eq-momentum-conserv-1})  to $A(T) \vec{\tilde{u}}=0$, equivalent to $\vec{\tilde{u}}=0$. Exact expression for $A$ will be given in the next section.

Under the assumption of a small variation of the density and the temperature, the Boussinesq approximation allows to linearize the density in the buoyancy part $\rho g$ of the eq. (\ref{eq-momentum-conserv-1}) as follows:
\begin{equation}
   \rho = \rho_{ref} (1 - \beta (T-T_{ref})),
\end{equation}
with $\beta = - (1/\rho_{ref}) (\partial \rho / \partial T)$.
Therefore, the momentum equation can be written as:

\begin{equation}\label{eq-momentum-conserv}
  \frac{\partial \vec{\tilde{u}}}{\partial t_{\varphi}} +   {(\vec{\tilde{u}}\cdot\nabla ) \vec{\tilde{u}}} + \nabla p - \nu_{l}  {\nabla^2 \vec{\tilde{u}}} 
- f_B(T) \vec{e}_y= A(T) \vec{\tilde{u}},
\end{equation}
where $p = (P + \rho_{ref} g y)/ \rho_{ref}$ and $f_B(T) = g \beta (T-T_{ref})$ denotes the buoyancy force.

Finally, the conservation of mass in the liquid phase is expressed by the continuity equation:
\begin{equation}\label{eq-mass-conserv}
\nabla \cdot \vec{\tilde{u}} = 0.
\end{equation} 


\section{Final system of equations for the single-domain approach}\label{sec-eq-scaling}

It is convenient to numerically solve a dimensionless form of the previous equations.
Using the cavity height $H$ as length scale and a reference state $(\rho, V_{ref}, T_{ref})$, we can define the following scaling for the space, velocity, temperature and time variables:
\begin{equation}\label{eq-adim}
\vec{x} = \frac{\vec{\tilde{x}}}{H} \, , \,  \vec{u} = \frac{\vec{\tilde{u}}}{V_{ref}} \, , \,  \theta = \frac{T-T_{ref}}{\delta T} \, , \,  t = \frac{V_{ref}}{H} \, t_{\varphi},
\end{equation}
Temperatures $T_h$ (hot) and $T_c$ (cold) will be used to set isothermal walls of the cavity. The difference $\delta T$, 
%$\delta T=T_{h}-T_{f}$with $T_f$ the temperature of fusion, 
is considered as the representative temperature scale  for the natural convection onset in the liquid region. 
For the natural convection problem without phase-change, $\delta T$ could be defined as $\delta T=T_{h}-T_{c}$ since the flow in the fluid is driven by the temperature difference between the "hot" and the "cold" temperature.
However, for the melting PCM, the convection is driven by the temperature difference $\delta T=T_{h}-T_{f}$, with $T_f$ the temperature of fusion.
As far as the solidification process is concerned, a distinct discussion will be provided in section \ref{chap-SOLIDIFICATION}. % during the melting,  and $\delta T = T_f - T_c$ during the solidification. }
Thus $\delta T$ is used to define the Rayleigh number of the flow:
\begin{equation}
\label{eq-Rayleigh}
\Ray = \frac{g \beta H^3 \delta T}{\nu_l \alpha_l},
\end{equation}
where $\alpha = k/(\rho c)$ is the thermal diffusivity and  $\beta$ the thermal expansion coefficient. 
Note that the reference temperature for the phase-change problem is   $T_f$, resulting in  $\theta_f = 0$.
Since the mushy zone is defined for  $  \theta_f - \varepsilon1 \, \leq \theta \leq \, \theta_f + \varepsilon2 $, this choice of the reference temperature 
simplifies the identification of the latter to  $  -\varepsilon \, \leq \theta \leq \,\varepsilon $.

Finally, the dimensionless system of equations to be solved in both liquid and solid regions can be written as:
\begin{eqnarray}
\nabla\cdot \vec{u}&=&0, \label{eq-qmvt} \\ \vspace{0.2cm}
 \frac{\partial \vec{u}}{\partial t} + {(\vec{u}\cdot\nabla) \vec{u}} +\nabla p -\frac{1}{\Rey}{\nabla^2 \vec{u}} 
 - f_B(\theta)\, \vec{e}_y - A(\theta) \vec{u}&=&0, \label{eq-qmvt-2} \\ \vspace{0.2cm}
 \frac{\partial \left(C \theta\right)}{\partial t} + \nabla \cdot\left( C \theta \vec{u}\right) -
 \nabla \cdot\left( \frac{K}{\Rey \Pr} \nabla \theta \right) +  \frac{\partial \left(C S\right)}{\partial t}  &=& 0, \label{eq-energ} 
\end{eqnarray}
where the linearised (Boussinesq) buoyancy force ($f_B$), the Reynolds ($\Rey$) and Prandtl ($\Pr$) numbers are defined as:
\begin{equation}\label{eq-RePr}
f_B(\theta) = \frac{\Ray}{\Pr \Rey^2} \theta, \quad \Rey = \frac{\rho V_{ref} H}{\mu_l}=  \frac{V_{ref} H}{\nu_l} , \quad \Pr = \frac{\nu_l}{\alpha_l}.
\end{equation}
Non-dimensional conductivity and specific heat are functions of the temperature $\theta$, 
\begin{equation}\label{eq-adimKC}
K(\theta)= \frac{k}{k_l} , \,  \, C(\theta) = \frac{c}{c_l},
\end{equation}
and have to take into account the variation of material properties between the solid and the liquid regions. 

In the energy equation (\ref{eq-energ}), the non-dimensional function $S = s/\delta T$, introduced by the enthalpy model, is regularized across the regularization region using a hyperbolic-tangent function \citep{dan-2014-JCP}:
\begin{equation}
S(\theta) = S_{l} + \frac{S_{s}-S_{l}}{2}\left\{
1 + \tanh\left(\frac{\theta_r-\theta}{R_r}\right)
\right\},
\label{eq-Stanh}
\end{equation} 
where $\theta_r$ is the central value around which we regularize (typically $\theta_r=\theta_f=0$) and $R_r$ the smoothing radius (typically $R_r=\varepsilon$). Note that $S_{s} = 0$ and
\begin{equation}
S_{l} = \frac{h_{sl}/c_l}{\delta T} = \frac{1}{\Ste},
\label{eq-Ste}
\end{equation} 
where $Ste$ is the Stefan number. Regularizations similar to (\ref{eq-Stanh}) are used to model the variation inside the regularization region of functions (\ref{eq-adimKC}) defining material properties.

Finally, the penalty term in the momentum equation (\ref{eq-qmvt-2}) is derived from the Darcy's law, by modeling the fluid flow within the mushy region as a flow through a porous medium.
In fact, the Darcy's law states that the velocity of flow in porous medium is proportional to the pressure gradient:

\begin{equation}
	\vec u = - \frac{\zeta^*}{\mu} \vec \nabla p.
\end{equation}
where $\zeta^*$ is the permeability, which is a function of the porosity.
As the porosity decreases, the permeability (and the velocity) also decreases, down to the limiting value of zero when the mush becomes completely solid.
This behavior can be accounted in a numerical model by adding a source term $A \vec u$ in the momentum equation.
The well-known equation derived from the Darcy law, the Carman-Kozeny equation \ref{ck eq}, could be a suitable form for the function $A$:

\begin{equation} \label{ck eq}
	\nabla p = - \frac{\CKC (1 - \lambda)^2}{\lambda^3} \vec u.
\end{equation}

Finally $A$ takes the form \citep{Belhamadia2012,kheirabadi2015effect}:

\begin{equation}\label{eq-CK}
A(\theta) = -\frac{\CKC (1 - \varphi(\theta))^2}{\varphi(\theta)^3 + b},
\end{equation}
where $\varphi(\theta)$ is the phase-change variable, which is  $1$ in the fluid region and $0$ in the solid. Inside the regularization region,  $\varphi(\theta)$ is regularized using a hyperbolic-tangent function similar to (\ref{eq-Stanh}).
The constant $\CKC$ is set to a  large value (as discussed below) and  the constant $b=10^{-6}$ is introduced to avoid division by zero.

