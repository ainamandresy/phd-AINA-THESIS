%%%%%%%%%%%%%%%%%%don't forget if needed %%%%%%%%%%%%%%%%%%%%%
%\section[toc version]{title version%
%              \sectionmark{head version}}
%\sectionmark{head version}
%%%%%%%%%%%%%%%%%%%%%%%%%%%%%%%%%%%%%%%%%%%%%%%%%%%%%%%%%%%%%%
\def\titcourt{Introduction}
\def\titlong{Introduction}
%%%%%%%%%%%%%%%%%%%%%%%%%%%%%%%%%%%%%%%%%%%%%%%%%%%%%%%%%%%%%%%%
\chapter[\titlong]{\titlong%
              \chaptermark{\titcourt}}
\chaptermark{\titcourt}
%%%%%%%%%%%%%%%%%%%%%%%%%%%%%%%%%%%%%%%%%%%%%%%%%%%%%%%%%%%%%%%%
%%%%%%%%%%%%%%%%%%%%%%%%%%%%%%%%%%%%%%%%%%%%%%%%%%%%%%%%%%%%%%%%

\section{Phase change materials}

Interest in renewable and sustainable energy sources is increasing nowadays. 
In the past decades, the world's energy consumption was fully dominated by fossil fuels such as coal, natural gas or raw oil.
Today, we must face to the fact that these fossil fuels are going to an end.
Moreover, their large utilization has induced considerable impacts on our environment.
Notwithstanding until today, our dependency on those energy sources is still causing severer environmental issues.
Climate changing, mostly due to the huge CO2 and other green house gases released into the atmosphere, is more than ever experienced in our daily life:
unnatural heat waves, non-seasonnal precipitation, seal level rising, increasing melt of sea ice, deterioration of ozone layer, etc.
Thus, interest in environment-friendly energy sources, such as biomass, wind or solar have arisen.

Furthermore, even though solar or wind energy are now operational, their main drawback remains they availability that not necessarily match to consumers demand.
Solar energy is not available during the nights for example, while wind energy is intermittent over the year.
Yet, the energy demand varies with time and the energy suppliers have to meet this demand.
An efficient solution to this discrepancies of availability and demands problem is the use of energy storage systems.
The idea is to store the energy at one time in one forme or another and release it latter for a particular need, since energy availability and demand rarely concur.
The energy storage technologies are hence an essential point in the domain of the alternative energy.
The later can be divided into five classes:
magnetic, biological, chemical, mechanical and thermal energy storages.
The main feature of the aforementioned energy storage technologies relies on energy charging and discharging process.
In most cases however, thermal energy is the energy form widely used.
Even the electricity generation is monitored by heat generating high temperature and high pressure.
Storing thermal energy is thus an efficient and fundamental way to store the energy.
It can be realised by rising the substance's temperature (sensible heat energy storage) or by changing the substance's phases (latent heat energy storage).

Research on passive heat storage system have attracted many considerations lastly, namely interest on Phase Change Material (PCM) as latent heat energy storage have arisen.
PCM are used to store heat during the melting of the materials and released later the stored heat during the solidification process.
At present, the latent heat storage technologies are proven as an effective solution to decrease the use of fossil fuel and in the same time increase the energy usage efficiency.

Moreover, taking advantage of the high value of the latent heat of solidi-liquid transformations, PCMs are also widely encountered in thermal regulation in buildings to reduce overheating.
In summer, PCMs are used to absorb the excessive solar radiation heat and maintain a bracing indoor ambience.
During winter however, PCMs can be used to store heat generated by electrical heating system during the night and then release it in the daytime.
In fact, latest announcement of the french ministry of ecological transition, detailing the repartition of the energy consumption in different domains, indicates more than $35 \%$ of the total energy being consumed by residentials and commercial buildings. Details about energy consumptions in 2016 are shown in Table 1.
This important energy demand in buildings is mostly due to the increasing need of comfort conditions and the associated market penetration of more cooling/heating systems.
More than $60\%$ of the total energy consumption in residential sector is dedicated to space heating.
Research towards energy-efficient building is thus rising to reduce heating and cooling demand.
The use of efficient insulation is the key of energy conservation in residential buildings.

%
%Particularly, storing the energy and reducing the building energy consumption have attracted many considerations.
%
%The main type of energy source used in the worls is now the traditionnal fossil fuel, that unfortunately causes environmental disaster.
%Moreover, fossil fuel sources ted to disappear in a couple of decades.
%The energy storage technology is an effective way to alleviate the dependency on fossil energy.
%The principle is to store the energy at a given time and release it latter for a particular need.
%The essential feature is that energy storage is an efficient way to deal with the energy management issue.
%The energy demand vary indeed with time and the energy suppliers have to meet this demand.
%Managing the energy availability and the energy demand is a real problem since they rarely concur.
%Solar energy is not available during the nights for example, while the wind energy production is intermittent over the year.
%Thus, storing efficiently the energy is essential.
%Many technologies are available to ensure this charging and discharging process of the energy.
%These technologies could be classified into five groups:\\
%(i) magnetig storage: capacitors, superconducting magnectic,\\
%(ii) biological storage,\\
%(iii) chemical storage: electrochemical (batteries), thermochemical,\\
%(iv) mechanical energy storage: kinetic energy, potential energy,\\
%(v) thermal energy storage: sensible heat storage, latent heat storage.\\
%The advantage of the thermal energy storage is the possible use of passive systems.
%Passive in the sens of no artificial or commercial energy (electricity, ...) is needed.
%Among of this is the use of phase change material taking advantage of high latent heat.

%Latest announcement of the french ministry of ecological transition draws the energy consumption repartition in different domains.
%Three main sectors principally consume the most energy.
%$30\%$ of the total energy were consumed by residential and commercial buildings, and transportations in 2016.
%The detailed energy consumptions are shown in Table 1.
%This important energy demand in buildings is mostly due to the increasing need of comfort conditions and the associated market penetration of more cooling/heating systems.
%More than $60\%$ of the total energy consumption in residential sector is dedicated to space heating.
%Research towards energy-efficient building is thus rising to reduce heating and cooling demand.
%The use of efficient insulation is the key of energy conservation in residential buildings.

\section{Purpose of the thesis}
\subsection{Existing method}
Solid-liquid phase-change is known to be a tough problem, be it theoretically or numerically.
The non-linear time-dependent evolution of the interface would raise for example a major issue.
Indeed, the strong coupling between the temperature and the velocity field makes the calculation of the position and the velocity of the interface very difficult.
From a mathematical point of view, actual chalenge is about the derivation of a veracious theoretical framework, or transport equations for different involved quantities (velocity, temperature, concentration, density, etc.).
As far as the one-dimensional Stefan problem is concerned, the existence and the uniqueness of the solution was proven \citep{rubinstein1947solution,evans1951note,douglas1957uniqueness}.
Moreover, exact solution for non-isothermal transitions and multi-dimensional problems were discussed by \citep{elliott1987error,cho1969heat} and both analytical and approximated solutions were given by \citep{tarzia2011explicit}. 
Furthermore, for the progressive consideration of the natural convection in phase-change problems, one can see the historical review by \citep{yao1989melting}. \\
Concerning the numerical modeling, the existing method in the literature can be divided into two main categories:
multi-domain or single-domain methods.
Each methods are appropriate for either pure fluid (pure material) or multi-phase (homogeneous mixtures) problems. \\
The multi-domain methods can be classified into two families:
the moving grid or the eulerian approaches. 
The front tracking methods, the mapping methods, the transformed grid methods or the front fixing methods belong to the moving grid families \citep{sparrow1977analysis,gupta2000moving}.
The main idea is to track explicitly the interface and thus to solve separately the liquid and the solid phases.
The interface is governed by energy balance, corresponding to a boundary condition for each phases. 
However, the eulerian methods aim to follow implicitly the interface by the mean of a new equation used to reconstruct its position.
The front fixing methods or phase fields methods are examples of such approaches.
The essential feature is that the interface is diffuse, 
at the example of the phase field methods (PFM). 
In PFM the interface is identified through a phase field variables $\phi$, involving in a free energy equation which is minimized. For a comprehensive review of these methods, see \citep{fix1982phase,davis2001theory,boettinger2002phase,singer2008phase}.\\
In contrast with multi-domain methods, single-domain methods (or enthalpy methods) solve a unique system of equations throughout the domain. 
Now the interface is neither track explicitly, nor implicitly, but is instead considered a posteriori from the computed temperature field.
The main feature of the method is about the formulation of the energy equation, which can be separated into two classes:
the enthalpy-based formulations or the temperature-based formulations.
First, the enthalpy-based formulations \citep{eyres1946calculation,rose1960method,bhattacharya2014enthalpy} aim at solving the enthalpy from the usual energy equation, and then from outer iterations until convergence, the temperature is updated from a temperature-enthalpy coupling model. 
A second variety consists of rewriting the energy equation with enthalpy terms only \citep{rady1996natural,hannoun2003resolving}.
Second, the temperature-based formulations consider the enthalpy as the sum of the latent and the apparent heats in the energy equation. 
The latent heat can be thus treated either by heat capacity coefficient methods \citep{gau1984melting,szekely1970effect,chiesa1974natural} or by heat source term methods \citep{Tenchev2005,swaminathan1997towards,dan-2014-JCP}.
The advantages and the drawbacks of each approaches were discussed in detail by \cite{konig2017comprehensive}.
By comparing several fixed-grid methods, they have concluded that enthalpy based formulations using the usual energy equation and the temperature-based formulations using heat source term approach were the more accurate and robust methods. 

It is worth noting that two other methods very different from the previous ones also exist in the literature:
the lattice boltzmann methods and the meshless methods.
The lattice boltzmann methods are derived from the lattice gas automata model: the behaviour of the particles are governed by a kinetic model, which is averaged to obtain macroscopic behaviour \citep{frisch1986lattice,luo2015lattice,gong2015numerical}.
The meshless methods use an arbitrary number of nodes, without mesh as suggested by their names, in which interpolations are done to approximate the solution. A more complete review is given by \cite{atluri2002meshless}.


\subsection{Present numerical method}
In this thesis, we use an enthalpy method with a temperature-based formulation using heat source term approach to simulate phase-change systems with convection.
The natural convection flow in the liquid phase is simulated by solving the full incompressible Navier-Stokes equations with Boussinesq approximation.
A Carman-Kozeny type penalty model is applied to ensure a zero velocity value in the solid region.
The main feature of our numerical approach is the use of an adaptive finite element method to accurately track the solid-liquid interface.
Single domain method requires actually a fine mesh near the phase change front in order to capture the large enthalpy gradient. 
The smaller the phase change interval the narrower the mushy region and the more refined the mesh should be.
Yet applying a fine mesh in the whole domain would increase considerably the computational time.
Hence, we introduce a FE method with time-dependent mesh adaptivity by metric control that is  effective for a large range of phase-change systems with convection, from melting to solidification. The proposed mesh refinement strategy has the capacity to take into account different metrics and thus the ability to refine the mesh in different regions of interest in the computational domain. In particular, we show that the method is able to simultaneously track several interfaces in the domain.\\ %, a feature that was not present in previous mesh refinement algorithms.\\
The nonlinear system of equations are solved by means of a Newton algorithm.
A fully-implicit Newton method for the phase-change system based on a finite-element formulation of the Navier-Stokes equations has been derived by \citep{dan-2014-JCP}.
The advantage of this formulation is to  permit a straightforward implementation of different types of non-linearities in the system of equations.

 Numerical methods based on standard finite elements are less represented in this field. To the best of our knowledge, no finite-element programs exist in the CPC Program Library for the phase-change systems simulation. The purpose of this paper is thus to distribute a finite-element solver for computing the 2D incompressible Navier-Stokes equation with Boussinesq approximation.
 The code was built as a toolbox for FreeFem++ \citep{hecht-2012-JNM,freefem}, which is a free software (under LGPL license) using a large variety of triangular finite elements  (linear and quadratic Lagrangian elements, discontinuous P$_1$, Raviart-Thomas elements, etc.)  to solve partial differential equations. FreeFem++ is an integrated product with its own high level programming language and a syntax close to mathematical formulations, making the implementation of numerical algorithms very easy. Among the features making FreeFem++ an easy-to-use and highly adaptive  software we recall the advanced automatic mesh generator, mesh adaptation, problem description by its variational formulation, automatic interpolation of data, colour display on line, postscript printouts, etc. FreeFem++ community is continuously growing, with  thousands of users all over the world.

\section{Thesis plan}
%
%The fundamental operational mode of latent thermal energy storage (LTES) systems based on phase-change materials (PCM) is made of alternate melting and solidification cycles that  are not necessarily periodic. Partial melting and/or solidification of the PCM are often observed in applications and, in particular, in applications for buildings \citep{zhu2009dynamic,ascione2014energy}. 
% Using the total latent heat storage potential offered by the PCM in energy storage requires  a complex design process. This could benefit from accurate numerical simulations of such incomplete charging/discharging cycles. 
%A wide range of recent applications is concerned by such modelling issues, including thermal energy storage (\eg for solar power generation) and passive temperature control (\eg for modern portable electronics) devices. For a review of various applications of PCMs with different
%melting temperatures in thermal energy storage systems, see recent reviews by \cite{agyenim2010review} and \cite{kalnaes2015phase}.
%
%
%Actual challenges in the  mathematical and numerical description  of a melting-solidifica\-tion cycle include i) the derivation of a realistic theoretical framework, using transport equations for different involved quantities (velocity, temperature, viscosity, density) and ii) the design of robust, accurate  and efficient numerical methods for solving these equations, for different initial and boundary conditions.
% 
%\Blue{As far as point i) is concerned, %\Red{prior to \cite{voller1987pcm},} most of the models consider the conduction as the principal mechanism in describing the heat transfer during melting or solidification (Stefan problem). 
%the solution of the Stefan problem has been provided by \cite{rubinstein1947solution}. }
%Later,  other important physical phenomena have been accounted for: gravity effects, convection in the liquid phase, the presence of a mushy region containing both solid and liquid parcels at the interface between the two phases, etc. 
%\Blue{For a comprehensive review of these approaches, see  \cite{kowalewski2004phase} and   \cite{faghri-rev2006}. 
%A historical review of the role played by convection in phase-change problems is provided by e.g.,  \cite{yao1989melting}.} 
%In particular, the natural convection in the liquid was proved to play an important role in the heat transfer between phases and in the propagation of the melting/solidification front \citep{Morgan1981,Voller1987,jany1988scaling,Evans2006,Vidalain2009,Wang2010}. As a consequence, modern simulations of phase-change systems are dealing with the Navier-Stokes equations for the liquid phase, using the Boussinesq approximation for thermal effects. 
%
%Single domain approaches are very convenient for numerical implementation, since the same system of Navier-Stokes-Boussinesq equations is solved inside both liquid and solid phases. Two ingredients are necessary to make possible the use of the single domain model.  
%First, the velocity inside the solid phase has to be set to zero.  
%This is achieved by directly setting the velocity to zero in finite-volume methods (\eg, \cite{Wang2010,wang2010numerical}) or by using penalty models in finite-elements methods, based on viscosity \citep{dan-2014-JCP} or Carman-Kozeny terms \citep{Belhamadia2012,zhang2015numerical,mencinger2004numerical}. 
%Second, the energy equation is written using an enthalpy-based model  \citep{voller1987pcm,Cao1989,Cao1990}. 
%The feature of the enthalpy method is its capability to deal with both mushy and single point phase changes.
%Indeed, in case of non-isothermal phase change, a mushy zone between the liquid and the solid phases characterizes the system.
%In case of pure materials, the phase change occurs at a fixed temperature; however an artificial mushy-zone is introduced between the solid and the liquid parts, just to regularize the enthalpy and other discontinuous parameters.
%
%
%The challenge for numerical systems solving the single-domain Navier-Stokes-Boussinesq model is to accurately capture the moving solid-liquid interface. The problem is even more challenging when several melting-solidification fronts, with distorted shapes are present (\eg the solidification after a partial melting). 
%When fixed uniform meshes are used, which is the case of a great majority of existing finite-volume codes, the grid density has to be considerably increased in the entire domain, making the simulation very expensive.   When a trade-off between accuracy and computational cost is sought, the fixed grid approach allows to place only a few computational cells inside  the regularization region.
%
%Dynamical mesh adaptivity becomes in this context a valuable tool  to concentrate the grid refinement effort only in regions displaying high gradients of the computed variables (melting-solidification fronts, thermal or viscous boundary layers). For the classical two-phase Stefan problem, \cite{Belhamadia2004_S} suggested  an anisotropic mesh adaptation algorithm based on solution-dependent metrics. The authors extended  their algorithm for the three-dimensional simulation of the same problem  \citep{Belhamadia2004_3D} and showed that the use of locally adapted meshes with strong anisotropy proved to be very effective in reducing the number of computational nodes for such phase-change systems without convection. To simulate melting or solidification problems with convection,  \cite{dan-2014-JCP}  recently suggested a dynamical mesh adaptation algorithm based on metrics control and implemented with the FreeFem++ software \citep{freefem,hecht-2012-JNM}. The advantage of this adaptive finite-element method, which will be also used in the present study, is to make possible, with reasonable computational cost, the re-meshing of the computational domain at each time step. A very refined discretization of the  regularization zone between solid and liquid phases is thus obtained, while regions with low gradients are de-refined in order to balance the overall computational effort. 
%
%
%
%The previously mentioned modern numerical approaches were mostly applied to simulate separately melting or solidification problems and only recently for alternate melting and solidification cycles \citep{wang2010numerical}. However, cyclic or periodic, melting and solidification problems have attracted considerable attention in the literature. \cite{ho1993periodic} and \cite{voller1996cyclic} studied numerically periodic melting in a square enclosure. Recently, \cite{hosseini2014experimental} presented experimental studies for the melting and the solidification of a cylindrical PCM during a charging and discharging process and \cite{chabot2017solid} studied analytically the effect of an alternate heating and cooling in a cylindrical PCM, with periodic boundary conditions. 
%
%	
%The present contribution is scoped  to offer an accurate numerical description of the alternate melting and solidification of a typical PCM. We use a finite-element numerical system with second-order  accuracy in time and space to solve the single-domain model based on Navier-Stokes equations with Boussinesq approximation. The main advantage of our method is that the mesh adaptivity algorithm could be applied each time step. The mesh is thus dynamically refined with respect to velocity and temperature variables, allowing to accurately capture the interface between solid and liquid phases, the boundary-layer structure at the walls and the details of the unsteady convection cells in the liquid.  
%
%We simulate a typical PCM configuration represented by a differentially heated square cavity filled with an octadecane paraffin. This is a well-established benchmark documented experimentally by \cite{Okada1984} and extensively used  to validate numerical codes \citep{Wang2010,jany1988scaling,mencinger2004numerical}. First,  we simulate the melting phase  and use this case to validate our numerical system against experimental and previously reported numerical data. Second, we consider two operating cases for the solidification process.  
%In the first study case the solidification starts after a  complete melting of the PCM (liquid fraction of 95\%), while in the second case after a partial melting (liquid fraction of 50\%). All cases are analysed in detail by providing temporal evolution of solid-liquid interface, liquid fraction, Nusselt number and accumulated heat input. Different heat transfer regimes are identified  and explained using scaling correlation theory.  Several practical implications for the two operating modes are finally drawn.
%
