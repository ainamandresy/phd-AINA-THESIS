%%%%%%%%%%%%%%%%%%don't forget if needed %%%%%%%%%%%%%%%%%%%%%
%\section[toc version]{title version%
%              \sectionmark{head version}}
%\sectionmark{head version}
%%%%%%%%%%%%%%%%%%%%%%%%%%%%%%%%%%%%%%%%%%%%%%%%%%%%%%%%%%%%%%
\def\titcourt{Numerical resolution of the Navier-Stokes-Boussinesq model}
\def\titlong{Numerical resolution of the Navier-Stokes-Boussinesq model}
%%%%%%%%%%%%%%%%%%%%%%%%%%%%%%%%%%%%%%%%%%%%%%%%%%%%%%%%%%%%%%%%
\chapter[\titlong]{\titlong%
              \chaptermark{\titcourt}}
\chaptermark{\titcourt}
\label{chap-NSB}
%%%%%%%%%%%%%%%%%%%%%%%%%%%%%%%%%%%%%%%%%%%%%%%%%%%%%%%%%%%%%%%%
%%%%%%%%%%%%%%%%%%%%%%%%%%%%%%%%%%%%%%%%%%%%%%%%%%%%%%%%%%%%%%%%

%\section{Governing equations} \label{sec-gov-eq}
%%%%%%%%%%%%%%%%%%%%%%%%%%%%

We consider a solid-liquid system placed in a two-dimensional square cavity of height $H$. In the following, subscripts $s$ and $l$ will refer to the solid and liquid phases, respectively. 

For the numerical implementation, it is convenient to adopt a single-domain approach to describe both phases using the same system of equations. 
The model is based on the Navier-Stokes equations with Boussinesq approximation, which is the natural description of the fluid flow with natural convection. 
A penalty term is added to the momentum equations to bring the velocity to zero inside the solid region. 
For the energy conservation equation, an enthalpy method is used to model the phase change process. The single-domain model is described in detail in the following sections.

\section{Enthalpy method}

The phase change process is modelled using an enthalpy method \citep{voller1987pcm,Cao1989,Cao1990} with temperature-based formulation. We start from the energy equation:
\begin{equation}
\label{eq-energie}
   \frac{\partial (\rho h)}{\partial t_{\varphi}} + \nabla \cdot(\rho h \vec{\tilde{u}}) - \nabla \cdot (k \nabla T) = 0,
\end{equation}
where $t_{\varphi}$ is the physical time, $h$ the enthalpy, $\rho$ the density, $\vec{\tilde{u}}$  the velocity vector, $T$ the temperature and $k$ the thermal conductivity. 
To make Equation (\ref{eq-energie})  valid for the entire domain containing both liquid and solid phases, the total enthalpy $h$ is regarded as the sum of the sensible heat and the latent heat:
\begin{equation}
\label{eq-enth-model}
  {{h}} = c ( T + s(T) ),
\end{equation} 
with $c$ the local specific heat. The function $s(T)$ is introduced to model the jump of the enthalpy due to the phase change and is theoretically a Heaviside step function depending on the temperature: it takes the zero value in the solid region and a large value in the liquid, equal to $h_{sl}/c$, with $h_{sl}$ the latent heat of fusion. 
Linear  \citep{voller1987pcm,Wang2010} or smoother functions \citep{dan-2014-JCP} can be used to regularize $s(T)$ and also the jump of material properties (from solid to liquid). 
We use a regularization of all step-type functions by a continuous and differentiable hyperbolic-tangent function suggested by \cite{dan-2014-JCP} (see below). 
\Blue{%Moreover, we assume that the undercooling problem is here negligible since only pure and homogeneous materials are considered.
We assume moreover that the undercooling phenomenon is negligible (see also \cite{wang2010numerical,kowalewski2004phase}).}

Equation (\ref{eq-energie}) can be further simplified by considering the following assumptions: (i) the density difference between solid and liquid phases is negligible, \ie $\rho_l=\rho_s=\rho$ is constant; (ii) the regularization zone is narrow and the velocity inside this zone is very low. 
Consequently, the final expression of the energy equation is obtained by combining (\ref{eq-enth-model})  and (\ref{eq-energie}) and  neglecting the convection term $\nabla \cdot ( c s \vec{\tilde{u}})$\footnote{In the liquid phase, $\nabla \cdot ( c s \vec{\tilde{u}})  = h_{sl} \nabla \cdot  \vec{\tilde{u}}=0$; in the solid phase, $s=0$; in the regularization region, it is assumed that $\vec{\tilde{u}}=0.$}:
\begin{equation}\label{eq-energie-enth-model}
\frac{\partial \left(c T\right)}{\partial t_{\varphi}} + \nabla \cdot\left( c T \vec{\tilde{u}}\right) -
\nabla \cdot\left( \frac{k}{\rho} \nabla T \right) +  \frac{\partial \left(c s\right)}{\partial t_{\varphi}}  = 0.
\end{equation}
The essential feature of the current approach is that the phase change front is not tracked explicitly but is instead recovered a posteriori from the computed temperature field.
The phase-change occurs over a temperature interval $  T \in [T_f - T_{\varepsilon1}, T_f + T_{\varepsilon2}] $ around the temperature of fusion $T_f$.
For non-isothermal phase-change PCM, $T_{\varepsilon1}$ and $T_{\varepsilon2}$ correspond to the solidus and the liquidus temperature of the material.
However, for a pure material involving a unique phase-change temperature, $T_{\varepsilon1}$ and $T_{\varepsilon2}$ represent an artificial mushy zone used to regularize discontinuous parameters and should be set as small as possible.

\section{Navier-Stokes equations with Boussinesq approximation}
 

The natural convection in the liquid part of the system is modelled using the incompressible Navier-Stokes equations, with  Boussinesq approximation for buoyancy effects. To make this model valid for both liquid and solid phases, the momentum equation is modified as follows:
\begin{equation}\label{eq-momentum-conserv-1}
  \rho \left( \frac{\partial \vec{\tilde{u}}}{\partial t_{\varphi}} +   {(\vec{\tilde{u}}\cdot\nabla ) \vec{\tilde{u}}} \right) + \nabla P - \mu_{l}  {\nabla^2 \vec{\tilde{u}}} 
- \rho g \vec{e}_y= A(T) \vec{\tilde{u}},
\end{equation}
where $P$ denotes the pressure, $\mu_{l}$ the dynamic viscosity of the liquid (assumed to be constant).  
%and $f_B(T)$ the Boussinesq force. 

The penalty term $A(T) \vec{\tilde{u}}$ is artificially introduced in (\ref{eq-momentum-conserv-1}) to extend this equation in the solid phase, where the velocity, pressure, viscosity and Boussinesq force are meaningless.  Consequently, $A(T)$  is modelled to vanish in the liquid, where the Navier-Stokes-Boussinesq momentum equation is recovered. A large value of $A(T)$ is imposed in the solid, reducing the momentum eq. (\ref{eq-momentum-conserv-1})  to $A(T) \vec{\tilde{u}}=0$, equivalent to $\vec{\tilde{u}}=0$. Exact expression for $A$ will be given in the next section.

Under the assumption of a small variation of the density and the temperature, the Boussinesq approximation allows to linearize the density in the buoyancy part $\rho g$ of the eq. (\ref{eq-momentum-conserv-1}) as follows:
\begin{equation}
   \rho = \rho_{ref} (1 - \beta (T-T_{ref})),
\end{equation}
with $\beta = - (1/\rho_{ref}) (\partial \rho / \partial T)$.
Therefore, the momentum equation can be written as:

\begin{equation}\label{eq-momentum-conserv}
  \frac{\partial \vec{\tilde{u}}}{\partial t_{\varphi}} +   {(\vec{\tilde{u}}\cdot\nabla ) \vec{\tilde{u}}} + \nabla p - \nu_{l}  {\nabla^2 \vec{\tilde{u}}} 
- f_B(T) \vec{e}_y= A(T) \vec{\tilde{u}},
\end{equation}
where $p = (P + \rho_{ref} g y)/ \rho_{ref}$ and $f_B(T) = g \beta (T-T_{ref})$ denotes the buoyancy force.

Finally, the conservation of mass in the liquid phase is expressed by the continuity equation:
\begin{equation}\label{eq-mass-conserv}
\nabla \cdot \vec{\tilde{u}} = 0.
\end{equation} 


\section{Final system of equations for the single-domain approach}\label{sec-eq-scaling}

It is convenient to numerically solve a dimensionless form of the previous equations.
Using the cavity height $H$ as length scale and a reference state $(\rho, V_{ref}, T_{ref})$, we can define the following scaling for the space, velocity, temperature and time variables:
\begin{equation}\label{eq-adim}
\vec{x} = \frac{\vec{\tilde{x}}}{H} \, , \,  \vec{u} = \frac{\vec{\tilde{u}}}{V_{ref}} \, , \,  \theta = \frac{T-T_{ref}}{\delta T} \, , \,  t = \frac{V_{ref}}{H} \, t_{\varphi},
\end{equation}
Temperatures $T_h$ (hot) and $T_c$ (cold) will be used to set isothermal walls of the cavity. The difference $\delta T$, 
%$\delta T=T_{h}-T_{f}$with $T_f$ the temperature of fusion, 
is considered as the representative temperature scale  for the natural convection onset in the liquid region. 
For the natural convection problem without phase-change, $\delta T$ could be defined as $\delta T=T_{h}-T_{c}$ since the flow in the fluid is driven by the temperature difference between the "hot" and the "cold" temperature.
However, for the melting PCM, the convection is driven by the temperature difference $\delta T=T_{h}-T_{f}$, with $T_f$ the temperature of fusion.
As far as the solidification process is concerned, a distinct discussion will be provided in section \ref{chap-SOLIDIFICATION}. % during the melting,  and $\delta T = T_f - T_c$ during the solidification. }
Thus $\delta T$ is used to define the Rayleigh number of the flow:
\begin{equation}
\label{eq-Rayleigh}
\Ray = \frac{g \beta H^3 \delta T}{\nu_l \alpha_l},
\end{equation}
where $\alpha = k/(\rho c)$ is the thermal diffusivity and  $\beta$ the thermal expansion coefficient. 
Note that the reference temperature for the phase-change problem is   $T_f$, resulting in  $\theta_f = 0$.
Since the mushy zone is defined for  $  \theta_f - \varepsilon1 \, \leq \theta \leq \, \theta_f + \varepsilon2 $, this choice of the reference temperature 
simplifies the identification of the latter to  $  -\varepsilon \, \leq \theta \leq \,\varepsilon $.

Finally, the dimensionless system of equations to be solved in both liquid and solid regions can be written as:
\begin{eqnarray}
\nabla\cdot \vec{u}&=&0, \label{eq-qmvt} \\ \vspace{0.2cm}
 \frac{\partial \vec{u}}{\partial t} + {(\vec{u}\cdot\nabla) \vec{u}} +\nabla p -\frac{1}{\Rey}{\nabla^2 \vec{u}} 
 - f_B(\theta)\, \vec{e}_y - A(\theta) \vec{u}&=&0, \label{eq-qmvt-2} \\ \vspace{0.2cm}
 \frac{\partial \left(C \theta\right)}{\partial t} + \nabla \cdot\left( C \theta \vec{u}\right) -
 \nabla \cdot\left( \frac{K}{\Rey \Pr} \nabla \theta \right) +  \frac{\partial \left(C S\right)}{\partial t}  &=& 0, \label{eq-energ} 
\end{eqnarray}
where the linearised (Boussinesq) buoyancy force ($f_B$), the Reynolds ($\Rey$) and Prandtl ($\Pr$) numbers are defined as:
\begin{equation}\label{eq-RePr}
f_B(\theta) = \frac{\Ray}{\Pr \Rey^2} \theta, \quad \Rey = \frac{\rho V_{ref} H}{\mu_l}=  \frac{V_{ref} H}{\nu_l} , \quad \Pr = \frac{\nu_l}{\alpha_l}.
\end{equation}
Non-dimensional conductivity and specific heat are functions of the temperature $\theta$, 
\begin{equation}\label{eq-adimKC}
K(\theta)= \frac{k}{k_l} , \,  \, C(\theta) = \frac{c}{c_l},
\end{equation}
and have to take into account the variation of material properties between the solid and the liquid regions. 

In the energy equation (\ref{eq-energ}), the non-dimensional function $S = s/\delta T$, introduced by the enthalpy model, is regularized across the regularization region using a hyperbolic-tangent function \citep{dan-2014-JCP}:
\begin{equation}
S(\theta) = S_{l} + \frac{S_{s}-S_{l}}{2}\left\{
1 + \tanh\left(\frac{\theta_r-\theta}{R_r}\right)
\right\},
\label{eq-Stanh}
\end{equation} 
where $\theta_r$ is the central value around which we regularize (typically $\theta_r=\theta_f=0$) and $R_r$ the smoothing radius (typically $R_r=\varepsilon$). Note that $S_{s} = 0$ and
\begin{equation}
S_{l} = \frac{h_{sl}/c_l}{\delta T} = \frac{1}{\Ste},
\label{eq-Ste}
\end{equation} 
where $Ste$ is the Stefan number. Regularizations similar to (\ref{eq-Stanh}) are used to model the variation inside the regularization region of functions (\ref{eq-adimKC}) defining material properties.

Finally, the penalty term in the momentum equation (\ref{eq-qmvt-2}) is derived from the Darcy's law, by modeling the fluid flow within the mushy region as a flow through a porous medium.
In fact, the Darcy's law states that the velocity of flow in porous medium is proportional to the pressure gradient:

\begin{equation}
	\vec u = - \frac{\zeta^*}{\mu} \vec \nabla p.
\end{equation}
where $\zeta^*$ is the permeability, which is a function of the porosity.
As the porosity decreases, the permeability (and the velocity) also decreases, down to the limiting value of zero when the mush becomes completely solid.
This behavior can be accounted in a numerical model by adding a source term $A \vec u$ in the momentum equation.
The well-known equation derived from the Darcy law, the Carman-Kozeny equation \ref{ck eq}, could be a suitable form for the function $A$:

\begin{equation} \label{ck eq}
	\nabla p = - \frac{\CKC (1 - \lambda)^2}{\lambda^3} \vec u.
\end{equation}

Finally $A$ takes the form \citep{Belhamadia2012,kheirabadi2015effect}:

\begin{equation}\label{eq-CK}
A(\theta) = -\frac{\CKC (1 - \varphi(\theta))^2}{\varphi(\theta)^3 + b},
\end{equation}
where $\varphi(\theta)$ is the phase-change variable, which is  $1$ in the fluid region and $0$ in the solid. Inside the regularization region,  $\varphi(\theta)$ is regularized using a hyperbolic-tangent function similar to (\ref{eq-Stanh}).
The constant $\CKC$ is set to a  large value (as discussed below) and  the constant $b=10^{-6}$ is introduced to avoid division by zero.

%This chapter sets the mathematical and physical basis of the numerical system used to simulate the flow inside a cavity. To start with, we present the incompressible Navier-Stokes system of equations and introduce the Boussinesq approximation for buoyancy effects. The system of equations is then written in a non-dimensional form appropriate for numerical simulations. The numerical algorithm for solving this system of equations is described in detail (integration scheme,  finite-difference discretization, projection method and Poisson solver, boundary conditions). Finally, we present some basic theoretical tests to assess for the accuracy of the high-order finite-difference scheme used for the spatial discretization.
%
%
%\section{Motivation for the choice of the numerical method}
%%%%%%%%%%%%%%%%%%%%%%%%%%%%%%%%%%%%%%%%%%%%%%%%%%%%%%%%%%%%%%%%%
%
%
%Nowadays, there are so many in-house or commercial codes for simulating fluid flows or heat transfer phenomena. The main idea in developing this new code was to combine high-order methods, that are validated and popular in fluid dynamics community, to simulate natural convection (Boussinesq) flows. The reference benchmarks in this field are based on  simulations using spectral methods \citep{LeQuere-1987,LeQuere-1998,LeQuere91}. Spectral methods may become cumbersome when simulating configurations with obstacles (complex geometries) and non-standard boundary conditions, as is the case of outdoor telecommunication cabinets. Therefore, we have chosen to use high-order compact finite-difference methods offering spectral-like resolution and more flexibility in modeling immersed boundaries and non-linear boundary conditions. 
%
%
%Compact schemes are high-order implicit finite-difference schemes that have became very popular in fluid dynamics after the publication of the seminal paper by \cite{S.K.Lele-1992}. Lele used for the first time sixth order compact schemes for solving the compressible Navier-Stokes equations on a staggered grid and proved the spectral-like accuracy of the method. Since then, compact schemes  
%have been successfully applied to a large range of flow configurations from direct numerical simulation (DNS) of compressible \citep{S.S.Lele-1997,Mahesh-1997,Freund-2000} or incompressible flows \cite{Hussein-1994,Chu-1998,Verstappen-2003} to computational aeroacoustics \citep{Colonius-1997} and large eddy simulations (LES) \citep{Moin-1991,Constantinescu-2001}. 
%A large amount of literature exists in this field.
%
%The mesh arrangement of the computational nodes proved to be important for the accuracy of results. The collocated arrangement 
%(velocity and pressure are computed at the same location) introduces aliasing errors which could be larger than for explicit finite difference schemes \citep{Moin-1991,Constantinescu-2001}. 
%This effect was removed by the use of a staggered grid (velocity and pressure nodes are shifted) which lead to improved robustness \citep{Nagarajan-2003}. High order schemes on staggered meshes  were recently proposed for compressible \citep{Boersma-2005,Moore-2007} and incompressible \citep{Boersma-2011} Navier-Stokes equations. Compact Pad{\'e}-type schemes for other types of meshes were derived: stretched grids \citep{gamet-1999}, stretched, curvilinear, and deforming meshes \citep{Visbal-2002}.
%
%
%Most of the numerical methods used  in the numerical heat transfer community use  finite volume  methods.
%Classical second-order finite difference methods are also used, but 
%there are very few applications of high-order finite difference schemes for Boussinesq flows. 
%\cite{Ghade-2011} used a forth-order compact scheme for the vorticity-stream function-density formulation of Navier-Stokes-Boussinesq equations - this approach is limited to 2D flows.
%In this context, the novelty of our approach is to apply sixth order compact schemes for the 3D velocity-pressure description of Boussinesq flows. This high-order spatial discretization will be combined with high order integration schemes (third order Runge-Kutta) and accurate (TVD) methods for capturing the temperature evolution.
%
%
% 
%
%\section{Physical problem and Navier-Stokes-Boussinesq equations}
%%%%%%%%%%%%%%%%%%%%%%%%%%%%%%%%%%%%%%%%%%%%%%%%%%%%%%%%%%%%%%%%%
%
%The flow inside an outdoor telecommunication cabinet is submitted to unsteady thermal effects due to internal heat sources (electronic components) and exchanges with the external atmosphere (solar flux). These temperature gradients trigger inside the cabinet a main natural convection air flow, driven by the buoyancy force. For natural convection flows, a widely used simplification of the governing equations is the Boussinesq approximation: the fluid is supposed incompressible (the velocity vector field is divergence free) and density variations are taken into account only in the body-force (gravity) term of the momentum equations, through a linearized model. 
%
%More in detail, we present in the following the exact form of the equations together with the underlying simplifying hypothesis. The equations are non-dimensionalized using a length-scale  $L_\vref$ and a reference state $(\rho_\vref, V_\vref, T_\vref)$, defining the following scaling for the space, velocity, temperature and time variables:
%
%\begin{equation}
%\vec{x}=\frac{\vec{x}^{*}}{L_\vref}, \quad \vec{v}=\frac{\vec{v}^{*}}{V_\vref}, \quad \theta =\frac{T-T_\vref}{T_{h}-T_{c}}, \quad
%t=\frac{t^{*}}{t_\vref}, \quad t_\vref=L_\vref / V_\vref,
%\label{eq-num-adim}
%\end{equation}
%where {\em star} variables are in physical units and hot $T_h$ and, respectively, cold $T_c$ temperatures define the temperature gradient driving the natural convection flow. In this setting, the incompressible Navier-Stokes equations with the Boussinesq approximation can be derived as follows: 
%
%$\textbf{The continuity (mass conservation) equation}.$ The flow is supposed incompressible and thus:
%\begin{equation}
%\nabla\vva{v}^{*}=0 \quad \Longrightarrow \quad \nabla\vva{v}=0.
%\label{eq-num-cont}
%\end{equation}
%
%$\textbf{The momentum equation}.$ The model assumes that density variations around a reference state are very small and their effect on the inertia  terms are negligible. As a consequence, the variations of the density are taken into account only in the buoyancy (gravity) term of the Navier-Stokes momentum equations:
%\begin{equation}
%\rho_\vref\left[\frac{\partial\vva{v}^{*}}{\partial t}+(\vva{v}^{*}\,\nabla)\vva{v}^{*}\right]=-\nabla p^{*}+\Delta{\vva{v}}^{*} + \rho\vva{g}.
%\label{eq-num-mom-phys}
%\end{equation}
%The buoyancy therm is then linearized around the reference value $\rho_\vref$:
%\begin{equation}
%\rho\vva{g}=\rho_\vref\vva{g}+(\rho-\rho_\vref)\vva{g}, \quad \mbox{and} \quad (\rho-\rho_\vref)\vva{g}=-\rho_\vref\beta (T-T_\vref)\vva{g},
%\label{eq-num-mom-dens}
%\end{equation}
%where $\beta$  is the volume (bulk) expansion coefficient:
%\begin{equation}
%\beta=-\frac{1}{\rho_\vref}\left(\frac{\partial\rho}{\partial T}\right)_{p=constant}
%\end{equation}
%The momentum equation (\ref{eq-num-mom-phys}) becomes (since $\vva{g}= -g \vva{e}_z = -g \nabla z$):
%\begin{equation}
%\rho_\vref\left[\frac{\partial\vva{v}^{*}}{\partial t}+(\vva{v}^{*}\,\nabla)\vva{v}^{*}\right]=  -\nabla(p^{*}+\rho_\vref g z)+\Delta{\vva{v}}^{*} + \rho_\vref \beta (T-T_\vref){g} \vva{e}_z.
%\label{eq-num-mom-phys2}
%\end{equation}
%Denoting by $\bar{p}^{*}=p^{*}+\rho_\vref g z$ the kinetic pressure, we notice that the pressure $p^{*}$ of the system is the sum between the kinetic pressure and the hydrostatic pressure ($-\rho_\vref g z$). Since the hydrostatic component of the pressure has a simple universal form, we need to compute  only the kinetic portion of the pressure. We therefore use 
%in the equations the dimensionless pressure defined by $p=\bar{p}^{*}/(\rho_\vref V_\vref^2)$ and finally obtain the non-dimensional form of the momentum equation:
%\begin{equation}
%  \frac {\partial \vva{v}}{\partial t} +(\vva{v}\, \nabla)\vva{v}=-\nabla p +\frac{1}{Re}\triangle\vva{u}+\frac{\Ray}{\Rey^{2}\Prd}T \vva{e}_z.
%  \label{eq-num-mom-nond}
%\end{equation} 
%The non-dimensional (similarity) parameters of the flow are the Reynolds, Prandtl and Rayleigh numbers, defined as:
%\begin{equation}
%\Rey = \frac{\rho_\vref V_\vref L_\vref}{\mu}, \quad  \Prd = \frac{\nu}{\alpha}, \quad
%\Ray=\frac{g\beta L_\vref^3 (T_h-T_c)}{\nu \alpha}.
%\label{eq-num-numbers}
%\end{equation}
%Two other non-dimensional parameters are usually defined for natural convection flows, the Grashof and Bousinesq numbers: 
%\begin{equation}
%\Gra= \frac{g\beta L_\vref^3 (T_h-T_c)}{\nu^{2}} =\frac{\Ray}{\Prd}, \qquad \Bou=\Ray \Prd=\Gra \Prd^{2}.
%\label{eq-num-numbers2}
%\end{equation}
%We recall that in previous definitions, $\mu$ denotes the  viscosity, $\nu$ the kinematic viscosity, $\alpha$ the thermal diffusivity, $\beta$ the volume (bulk) expansion coefficient and $g$ the gravitational acceleration. We also recall the physical meaning of these parameters:
%\begin{itemize}
%\item[$\bullet$] the Reynolds number gives a measure of the ratio of inertial forces to viscous forces; for each flow, a critical value of the Reynolds number defines the transition between the laminar and turbulent regimes;
%
%\item[$\bullet$] the Prandtl number represents the ratio of momentum diffusivity (kinematic viscosity) to thermal diffusivity; it allows to evaluate the relative thickness of the momentum (velocity) and thermal boundary layers in heat transfer problems;
%
%\item[$\bullet$] the Grashof number represents the ratio of the buoyancy to viscous force acting on the fluid; it characterizes thermal effects in natural convection flows;  
%
%\item[$\bullet$] the Rayleigh number plays the role of a Reynolds number for  buoyancy driven flow (natural convection); in regard to it's critical value, it determines if the heat transfer is dominated by convection or conduction. 
%\end{itemize}
%
%
%
%
%
%$\textbf{The energy equation}.$  For the energy conservation equation, the Boussinesq approximation supposes that the interior forces are negligible, as well as the dependence of the internal energy with pressure.
%The thermodynamic coefficients are also presumed constants. These assumptions stand only when temperature variations  $\delta T$ are small: for air $\delta T < 15^{\circ}$ and for water $\delta T < 2^{\circ}$.
%As a consequence, the temperature is governed by a general {\em passive scalar} equation:
%\begin{equation}
%\frac{\partial T^{*}}{\partial t}+(\vva{v}\, \nabla)T^{*}=\left(\frac{\lambda}{\rho_\vref C_{p}}\right)\Delta T,
%\label{eq-num-energ1}
%\end{equation}
%where $\lambda$ is the thermal conductivity and $C_{p}$ the specific heat. The non-dimensional form of the energy equation follows since $\alpha={\lambda}/({\rho_\vref C_{p}})$:
% \begin{equation}
%\frac{\partial T}{\partial t}+(\vva{v}\, \nabla T)=\frac{1}{\Rey \Prd}\Delta T.
%\label{eq-num-energ2}
%\end{equation}
%
%
%
%$\textbf{Final equations}.$ We recall here the final system of equations to be solved:
%\begin{eqnarray}\label{eq-num-final1}
%\nabla\vva{v} &=&0,\\\label{eq-num-final2}
%\frac {\partial \vva{v}}{\partial t} +(\vva{v}\, \nabla)\vva{v}&=&-\nabla p +\frac{1}{Re}\triangle\vva{u}+\frac{\Ray}{\Rey^{2}\Prd}T \vva{e}_z, \\
%\frac{\partial T}{\partial t}+(\vva{v}\, \nabla T)&=&\frac{1}{\Rey \Prd}\Delta T.
%\label{eq-num-final3}
%\end{eqnarray}
%
%It is important to note that the characteristic scales in (\ref{eq-num-adim}) could be defined accordingly to the physics of the considered problem. Several choices are used in the literature:
%\begin{eqnarray}
%\label{eq-num-scal1}
%V_\vref^{(1)} = \frac{\nu}{L_\vref }&\Longrightarrow \ds t_\vref^{(1)} = \frac{\nu}{L_\vref ^2} &\Longrightarrow \Rey=1,\\
%\label{eq-num-scal2}
%V_\vref^{(2)} = \frac{\alpha}{L_\vref }&\Longrightarrow \ds t_\vref^{(2)} = t_\vref^{(1)} \Prd &\Longrightarrow \Rey =1/\Prd,\\
%\label{eq-num-scal3}
%V_\vref^{(3)} = \frac{\nu_l}{L_\vref } \sqrt{\frac{\Ray}{\Prd}}&\Longrightarrow \ds t_\vref^{(3)} = t_\vref^{(1)} \sqrt{\frac{\Prd}{\Ray}} &\Longrightarrow \Rey = \sqrt{\frac{\Ray}{\Prd}},
%\end{eqnarray}
%The last choice (\ref{eq-num-scal3}) is generally used for classical natural convection problems, while the first two choices are preferred for  melting/freezing problems \citep{Wang2010}.
%
%We shall use in the following the last scaling choice (\ref{eq-num-scal3}), setting $\ds \Rey = \sqrt{\frac{\Ray}{\Prd}}$, so the buoyancy term in (\ref{eq-num-final2}) is simplified to $ T \vva{e}_z$.
%
%
%\pagebreak
%
%
%\section{Numerical method}
%%%%%%%%%%%%%%%%%%%%%%%%%%%%%%%%%%%%%%%%%%%%%%%%%%%%%%%%%%%%%%%%%
%
%
%\subsection{Time integration method}
%
%Time integration is based upon a classical projection method \cite{chor} consisting in the following steps: 
%
%{\bf (A) Predictor step:} the momentum equations (\ref{eq-num-final2}) are advanced in time using an explicit treatment of the pressure through a classical integration scheme. Explicit schemes are preferred to facilitate a further MPI-parallel implementation of the code. First-order Euler, second-order Adams-Bashforth and third order Runge-Kutta methods are available in the code.
%
%At the end of this step, a non-divergence free velocity field $(\widetilde{u})$ is computed, following the general discretization for an integration scheme with $l$ steps :
%\begin{equation}
%\frac{\widetilde{u}_{c}^i-u_{c}^{i}}{\delta t}= \gamma_{i} H^{i}_{c}  +\rho_{i} H_{c}^{i-1}-\alpha_{i} G_{c} p^{i}
%\label{eq-num-RKi}
%\end{equation}
%where  $c=x,y,z$ denotes each spatial direction, $i=1,2,\cdots,l$ the substep of the method. 
%$G$ stands for the pressure gradient and $H$ regroups for convective, diffusive and buoyancy terms.
%
%A third-order Runge-Kutta method was adapted from the {\em fractional-step} method widely used in fluid dynamics computations \citep{rai91,verzjcp,orlandi-book}. The coefficients of the scheme ($i=1,2,3$) are
%\begin{equation}
%\begin{array}{ccc}\vspace{0.2cm}
%\ds \gamma_{1} =\frac{8}{15}, & \ds\gamma_{2} =\ds\frac{5}{12}, &\ds \gamma_{3} =\frac{3}{4}, \\
%\ds\rho_{1} =0, & \ds\rho_{2}=\ds -\frac{17}{60},& \ds\rho_{3}=- \frac{5}{12},
%\end{array}
%\label{eq-num-RKcoef}
%\end{equation}
%and $\alpha_{i}=\gamma_{i}+\rho_{i}$. We notice that $\rho_{1}=0$ allows the integration procedure to begin without the need of having previous values (self-starting scheme).
%
%
%For the second-order Adams-Bashforth scheme $l=1$ and $\gamma_1=3/2$, $\rho_1=-1/2$ and $\alpha_1=1$. This is not a self-starting scheme and needs a first Euler step to start the computation.
%
%For the first-order Euler backward scheme, $l=1$ and $\gamma_1=0$, $\rho_1=1$ and $\alpha_1=1$.
%
%All these schemes were implemented in a compact manner in the code, allowing to easily switch from one scheme to another. 
%
%
%{\bf (B) Projection step:} The projection equation 
%\begin{equation}
%u^{i+1}=\widetilde{u}^i-\alpha_{i}\cdot \delta t \nabla \phi,
%\label{eq-num-proj}
%\end{equation}
%introduces a supplementary pressure-related variable $\phi$. Since the velocity $u^{i+1}$ has to be divergence free field, we obtain from the previous relationship a Poisson  equation for $\phi$:
%\begin{equation}
%\Delta \phi = \frac{1}{\alpha_{i}\, \delta t} \nabla \widetilde{u}. 
%\label{eq-num-phi}
%\end{equation}
%Two methods were employed for solving the Poisson equation: firstly, a periodical case was considered, and secondly a non-periodical case was tested in order to obtain the three-dimensional cavity case.
%For the periodical case we use a FFT (Fast Fourier Transform) following the periodic direction (in our case $x$).  For the second case, without periodicity, a cosine series expansion following the same direction ($x$) was derived and implemented based on classical FFTs \cite[for details, see][]{dan-2008-rap-jetles}. Compared to the Fourier series development, the cosine expansion permits to simulate a closed 3D cavity, with no-slip wall conditions following the three directions. It also benefits from the performances of basic FFTs subroutines.
%
%In both cases the resulting 2D system is solved by a cyclical reduction method, with the use of the BLKTRI Fortran library: FISHPACK. The idea of the cyclical reduction is the recursive reduction of the linear system size by successively eliminating the odd order unknowns \cite[for details, see][]{ballestra}. The advantage of the method is to allow the use of a non uniform grid.
%For computational efficiency, the derivatives of the Laplacian operator are discretized using a second order centered scheme.
%
%{\bf (C) Corrector step:} After solving the Poisson equation (\ref{eq-num-phi}), the projection equation (\ref{eq-num-proj}) is used to compute the corrected divergence free field $u^{i+1}$. The pressure for the next step is obtained by subtracting (\ref{eq-num-RKi}) from the same  equation with implicit pressure term ($Gp^{i+1}$); using the projection equation (\ref{eq-num-proj}) we obtain the following update for the pressure:  
%\begin{eqnarray}
%p^{i+1}=p^{i}+\phi.
%\label{eq-num-phip}
%\end{eqnarray}
%
%{\bf (D) Temperature update:} The computed divergence free field is finally used to compute the temperature distribution from (\ref{eq-num-final3}). This equation is discretized following the same time integration scheme as for momentum equations. A particular care is devoted to the discretization of convective terms using a TVD (Total Variation Diminishing) scheme described in detail below.  
%
%
%
%\subsection{Spatial discretization}
%% % % % % % % % % % % % % % % % % % % % % % % % % % %
%
%
%The computational grid is rectangular (fig.\ref{fig-domaincalc3D111}), using Cartesian coordinates.
%The grid is generated separately following each space direction $(x,y,z)$ by choosing ether a uniform distribution or a stretched mesh. The grid refinement is used only with second-order finite-difference schemes. Compact sixth-order schemes require uniform meshes in all space directions.
%\begin{figure}[!htbp]
%\begin{minipage}{0.5\linewidth}
%\begin{center}
% {\includegraphics[width=\textwidth]{cap_3/mesh_3D.jpg}}
%  \end{center}
%  \end{minipage}
%\begin{minipage}{0.5\linewidth}
%\begin{center}
% {\includegraphics[width=\textwidth]{cap_3/mv.jpg}}
%  \end{center}
%  \end{minipage}  
% \caption{Three-dimensional computational domain (left) and refined mesh near the walls in the $(y,z)$ section (right).}
%\label{fig-domaincalc3D111}
%\end{figure}
%
%We consider a staggered grid with the scalar variables (pressure, temperature) computed in the cell center and the velocity components on the cell faces (see fig. \ref{fig-refgrid}). The advantage of the staggered grids is that it avoids the decoupling between pressure and velocity, and border pressure problems that may occur in the cases of collocated variables.  This type of grid was  first used by \cite{harlow-1965} in their MAC method. The staggered grid is efficient for the computations, it respects conservation properties and needs a small amount of memory space.
%\begin{figure}[!htbp]
%%\begin{minipage}{0.7\linewidth}
%\begin{center}
% {\includegraphics[width=0.9\textwidth]{cap_3/maillage_non_uniform_plan_y-z_3.jpg}}
%  \end{center}
%%  \end{minipage}
% \caption{Illustration of the staggered arrangement of variables following the $y$ direction.}
%    \label{fig-refgrid}
%  \end{figure}
%  
%  
%The three-dimensional computational domain $\Omega$ of dimension $L_{x}\times L_{y}\times L_{z}$ is discretized using $N_{1}\times N_{2}\times N_{3}$ grid points. 
%At each node $(i,j,k)$ we associate  the corresponding $(i,j,k)$ primary cell, defined by $\left[ x_c(i),x_c(i+1) \right] \times \left[ y_c(j),y_c(j+1) \right] \times \left[ z_c(k),z_c(k+1) \right]$ for  $i=1,...N_{1}-1;j=1,...N_{2}-1; k=1,...N_{3}-1$. For the case of a uniform grid
%\begin{eqnarray}\label{eq-num-grid-xc}
%x_c(i)=(i-1)\delta x, &i=1,...N_{1},  &\delta x={L_{x}}/{(N_{1}-1)},\\
%y_c(j)=(j-1)\delta y, &j=1,...N_{2},  &\delta y={L_{y}}/{(N_{2}-1)},\\
%z_c(k)=(k-1)\delta z, &k=1,...N_{3},  &\delta z={L_{z}}/{(N_{3}-1)}.
%\end{eqnarray}
%The secondary grid (for scalar variables) is defined by cell centers, located at coordinates:
%\begin{eqnarray}\label{eq-num-grid-xm}
%x_m(i)=\left[x_c(i)+x_c(i+1)\right]/2, &i=1,...N_{1}-1, \\
%y_m(j)=\left[y_c(j)+y_c(j+1)\right]/2, &j=1,...N_{2}-1, \\
%z_m(k)=\left[z_c(k)+z_c(k+1)\right]/2, &k=1,...N_{3}-1.
%\end{eqnarray}
%
%
%The grid refinement  procedure is based on analytical coordinate transforms. It can be used with second-order finite-difference schemes to increase the grid resolution in the vicinity of heated obstacles, or near the walls where strong gradients are present.
%Let us consider only the $y$ direction (as in fig. \ref{fig-refgrid}) and set a generic coordinate transform $y_c=f(\xi)$, where $\xi$ maps the unitary interval:
%\begin{eqnarray}
%\xi(j)=(j-1)\delta \xi, &j=1,...N_{2},  &\delta \xi={1}/{(N_{2}-1)}.
%\label{eq-num-grid-xi}
%\end{eqnarray}
%The function $f$ is a combination of hyperbolic-tangent functions with tuning parameters that allow to refine the mesh at different positions (the middle, one border or both borders of the interval $[0,L_y]$). We have used different analytical functions proposed by \cite{orlandi-book} with optimal parameters suggested by \cite{ballestra,bentethese}. Once the primary mesh is computed, we use (\ref{eq-num-grid-xm}) to determine the secondary mesh. 
%
%
%The derivatives for the stretched mesh will be computed using the chain rule. For example, the first derivative of a generic quantity $\psi$ with respect to $x$ will be computed as
%\begin{equation}
%\frac{\partial \psi}{\partial y}=\frac{d \xi}{dy}\frac{\partial \psi}{\partial \xi}=\frac{1}{\frac{dy}{d \xi}} \frac{\partial u}{\partial \xi}.
%\label{eq-num-dpsidxi}
%\end{equation}
%It was shown \cite{orlandi-book} \cite[see also][]{ballestra} that a discrete computation of the metrics ${dy}/{d \xi}$ reduces the approximation error, when compared to the use of their analytical formula. Consequently, we use for the metric evaluation the following second order schemes:
%\begin{eqnarray}
%\left(\frac{dy}{d\xi}\right)(j) \approx dcy(j)&=&\frac{y_c(j+1)-y_c(j-1)}{2\,\delta \xi},  \quad j=2,N_2-1,\\
%dcy(1)&=&\frac{y_c(2)-y_c(1)}{\delta\xi},\\
%dcy(N_2)&=&\frac{y_c(N_1)-y_c(N_2-1)}{\delta\xi}, 
%\label{eq-num-dcx}
%\end{eqnarray}
%and for cell centers:
%\begin{eqnarray}
%\left(\frac{dy}{d\xi}\right)_{j+\frac{1}{2}} \approx
%dmy(i)&=&\frac{y_c(i+1) -  y_c(i)}{\delta \xi},  \quad
%i=1,N_2-1.
%\label{eq-num-dmx}
%\end{eqnarray}
%We shall see in the following how these metrics are used in computing different terms of the Navier-Stokes-Boussinesq equations when second-order finite difference schemes are used. 
%
%
%
%
%\subsection{Discrete formulation of momentum equations: second and sixth-order schemes}
%
%We illustrate here the discretization of the momentum equations on the staggered grid. In the integration scheme
%(\ref{eq-num-RKi}), the explicit term $H$ contains the convective, diffusive and buoyancy terms:
% \begin{eqnarray}
% \label{eq-num-Hx}
% H_x =&\ds-\frac{\partial u^{2}}{\partial x}-\frac{\partial vu}{\partial y} -\frac{\partial wu}{\partial z}&+ \frac{1}{\Rey}\left( \frac{\partial^{2} u}{\partial x^{2}}+\frac{\partial^{2} u}{\partial y^{2}}+ \frac{\partial^{2} u}{\partial z^{2}}\right),\\  \label{eq-num-Hy}
% H_y =&\ds-\frac{\partial uv}{\partial x}-\frac{\partial v^{2}}{\partial y} -\frac{\partial wv}{\partial z}&+
%\frac{1}{Re}\left( \frac{\partial^{2} v}{\partial x^{2}}+\frac{\partial^{2} v}{\partial y^{2}}+ \frac{\partial^{2} v}{\partial z^{2}}\right), \\  \label{eq-num-Hz}
%H_z =&\ds-\frac{\partial uw}{\partial x}-\frac{\partial vw}{\partial y} -\frac{\partial w^{2}}{\partial z}&+
%\frac{1}{Re}\left( \frac{\partial^{2} w}{\partial x^{2}}+\frac{\partial^{2} w}{\partial y^{2}}+ \frac{\partial^{2} w}{\partial z^{2}}\right)+ \frac{\Ray}{\Rey^{2}\Prd} T.   
% \end{eqnarray}
%Following the staggered arrangement (fig. \ref{fig-refgrid}), these terms will be computed at different locations, corresponding to computational nodes for velocities. More precisely, we compute\\
%$H_x$  at $(i,j+1/2,k+1/2)$, \ie  $(x_c(i),y_m(j),z_m(k))$,\\
%$H_y$  at $(i+1/2,j,k+1/2)$, \ie  $(x_m(i),y_c(j),z_m(k))$,\\
%$H_z$  at $(i+1/2,j+1/2,k)$, \ie  $(x_m(i),y_m(j),z_c(k))$.\\
%The staggered arrangement implies the use of interpolation of variables, with different procedures depending on the order of finite-difference scheme.
%
%
%\subsubsection{Second-order centered schemes}
%% % % % % % % % % % % % % % % % % % % % % % %
%
%For a mesh generated by a coordinate transform, as described previously, we will need, following (\ref{eq-num-dpsidxi}), to interpolate and derive variables with respect to the uniform coordinate $\xi$. For a uniform mesh, the  interpolation operator for a centered position is
%\begin{equation} 
%\overline {\psi}(\xi_i)=\frac{1}{2}\left[\psi (\xi_i+\frac{\delta \xi}{2})+\psi (\xi_i-\frac{\delta \xi}{2})\right]=\psi_i+{\cal O}(\delta \xi)^2, \quad \psi_i=\psi (\xi_i),
%\end{equation}
%and for a shifted position
%\begin{equation} 
%\overline {\psi}^{�}(\xi_i)=\frac{1}{2}\left[\psi (\xi_i � \delta \xi )+\psi (\xi_i)\right]=\psi_{i � \frac{1}{2}}+{\cal O}(\delta \xi). 
%\end{equation}
%The first derivative is discretized following a centered second order scheme
%\begin{equation} 
%\psi_\xi(\xi_i)=\frac{1}{\delta \xi}\left[\psi (\xi_i +\frac{ \delta \xi}{2} )- \psi (\xi_i -\frac{ \delta \xi}{2})\right]= \frac{d\psi}{d\xi}\Big|_i+{\cal O}(\delta \xi)^2, 
%\label{eq-num-der1}
%\end{equation}
%and the second derivative operator will be obtain by two successive application of the first derivative:
%\begin{equation}  
%\psi_{\xi\xi}(\xi_i) =  \frac{1}{\delta \xi}\left[\psi_{\xi}(\xi_i+\delta \xi/2)-\psi_{\xi}(\xi_i-\delta \xi/2), \right]
%\label{eq-num-der2v1}
%\end{equation}
%leading to the well-known centered scheme:
%\begin{equation} 
%\psi_{\xi\xi(\xi_i) }=\frac{1}{\delta \xi^2}\left[\psi (\xi_i +\delta \xi)- 2\psi (\xi_i) -\psi (\xi_i- \delta \xi)\right]= \frac{d^2 u}{d \xi^2}\Big|_i+{\cal O}(\delta \xi)^2.  
%\label{eq-num-der2v2}
%\end{equation}
%
%It is important to note that for the evaluation of the second derivatives, the formula (\ref{eq-num-der2v1}) has better approximation properties on stretched meshes than (\ref{eq-num-der2v2}) \citep{orlandi-book} \cite[see also][]{ballestra}.
%We shall use in the following the formula (\ref{eq-num-der1})  for the first derivative and (\ref{eq-num-der2v1}) for the second derivative and illustrate the discretization principle by considering the terms of the momentum equation following the direction $y$ (since the arrangement of variables could be followed from fig. \ref{fig-refgrid}).
%
%$\bullet$ {Convective terms following the y-direction} computed at $(i+1/2,j,k+1/2)$:\\
%\begin{eqnarray}
%\nonumber
%\frac{\partial v^{2}}{\partial y}=\frac{1}{\delta \xi_y\, dcy(j)}\left[  \left( \frac{v_{i,j,k}+v_{i,j+1,k}}{2}\right)^{2}-\left( \frac{v_{i,j,k}+v_{i,j-1,k}}{2}\right)^{2}  \right]\\
%\nonumber
%\frac{\partial uv}{\partial x}=\frac{1}{\delta \xi_x\, dmx(i)}\left[ \frac{v_{i,j,k}+v_{i+1,j,k}}{2} \frac{u_{i+1,j,k}+u_{i+1,j-1,k}}{2}-\frac{v_{i,j,k}+v_{i-1,j,k}}{2} \frac{u_{i,j,k}+
%\ds u_{i,j-1,k}}{2}  \right]\\
%\nonumber
%\frac{\partial wv}{\partial z}=\frac{1}{\delta \xi_z\, dmz(k)}\left[ \frac{v_{i,j,k}+v_{i,j,k+1}}{2} \frac{w_{i,j,k+1}-w_{i,j-1,k+1}}{2}-\frac{v_{i,j,k}+v_{i,j,k-1}}{2} \frac{w_{i,j,k}+
%\ds w_{i,j-1,k}}{2} \right]
%\nonumber
% \end{eqnarray} 
%where $\delta\xi_x=1/(N_1-1), \delta\xi_y=1/(N_2-1), \delta\xi_z=1/(N_3-1)$ and  $dmx(i), dcx(i)$, $dmy(j), dcy(j)$, $dmz(k), dcz(k)$ are 
%the metrics defined previously.
%
%Note that the uniform grid discretization is recovered when all the metrics are equal to 1.
%
%$\bullet$ {Diffusive terms following the y-direction} computed at $(i+1/2,j,k+1/2)$:\\
% \begin{eqnarray}
% \nonumber
% \frac{\partial^{2} v}{\partial y^{2}}=\frac{1}{\delta \xi_y\, dcy(j)}\left[ \frac{v_{i,j+1,k}-v_{i,j,k}}{\delta \xi_y\, dmy(j)}-\frac{v_{i,j,k}-v_{i,j-1,k}}{\delta \xi_y\, dmy(j-1)} \right]\\
% \nonumber
%\frac{\partial^{2} v}{\partial x^{2}}=\frac{1}{\delta \xi_x\, dmx(i)}\left[ \frac{v_{i+1,j,k}-v_{i,j,k}}{\delta \xi_x\, dcx(i+1)}-\frac{v_{i,j,k}-v_{i-1,j,k}}{\delta \xi_x\, dcx(i)} \right]\\
%\nonumber
%\frac{\partial^{2} v}{\partial z^{2} }=\frac{1}{\delta \xi_z\, dmz(k)}\left[ \frac{v_{i,j,k+1}-v_{i,j,k}}{\delta \xi_z\, dcz(k+1)}-\frac{v_{i,j,k}-v_{i,j,k-1}}{\delta \xi_z\, dcz(k)} \right]
%\nonumber
% \end{eqnarray}
%
%
%
%
%
%
%\subsubsection{Sixth-order compact schemes}
%% % % % % % % % % % % % % % % % % % % % % % % % % % % % % % % %
%
%Compact schemes are based on implicit relationships between the discrete values of derivatives. These values are computed for all grid points in one direction by inverting a linear system. The main advantages of these schemes are their spectral-like behavior (no numerical dissipation and good spectral resolution) and a better representation of small scales, when compared to classical explicit schemes (usually  second or fourth order centered schemes). 
%
%
%We consider for this study compact schemes for uniform grids \cite[for streched grids, see for example][]{gamet-1999}.
%The general idea in deriving compact schemes is to consider linear relationships between the values of the function and its derivative for a given stencil. Several families of compact implicit finite differences schemes were derived by \cite{S.K.Lele-1992} and became very popular because of the use of a three-point stencil only. For example, let us consider a function $f(x)$ of class $C^{\infty}$, discretized on a uniform grid of stepsize $h$ and denote by $f_i$ the value $f(x_i)$. 
%
%For the first derivative \cite{S.K.Lele-1992} considered  the general formula:
%\begin{eqnarray}\nonumber
%\beta f^{'}_{i-2}+\alpha f^{'}_{i-1}+f^{'}_{i}+\alpha f^{'}_{i+1}+\beta f^{'}_{i+2}=\\
%=c\frac{f_{i+3}-f_{i-3}}{6h}+b\frac{f_{i+2}-f_{i-2}}{4h}+a\frac{f_{i+1}-f_{i-1}}{2h},
%\label{eq-num-lele1d}
%\end{eqnarray}
%and derived the following linear system for the coefficients in order to obtain a sixth-order truncation error:
%\begin{eqnarray}
%a+b+c=1+2\alpha+2\beta    \\
%a+2^{2}b+3^{2}c=2\frac{3!}{2!}\left(\alpha+2^{2}\beta \right)   \\
%a+2^{4}b+3^{4}c=2\frac{5!}{4!}\left(\alpha+2^{4}\beta \right)
%\end{eqnarray}
%For $\beta=c=0$ a family of a three-point stencil schemes is obtained. For this study we have chosen the following scheme (largely used in the literature):
%\begin{equation}
%\beta=0,\quad  c=0,\quad  \alpha=\frac{1}{3},\quad  a=\frac{14}{9},\quad  b=\frac{1}{9},
%\end{equation}
%with the truncation error $T=\frac{4}{7!} h^{6} f^{(7)}$.
%
%For non-periodic meshes, different schemes are necessary for the nodes near the boundary. For the first  grid point  $i=1$, \cite{S.K.Lele-1992} considered for the first derivative a family of schemes of the form:
%\begin{equation}
%f^{'}_{1}+\alpha f^{'}_{2}=\frac{1}{h}\left( a f_{1} + b f_{2} + c f_{3} + d f_{4} \right).
%\end{equation}
%For $i=1$, we have chosen the third-order scheme with:
%\begin{equation}
% d=0,\quad  \alpha=2,\quad  a=-\frac{5}{2},\quad  b=2,\quad c=\frac{1}{2}.
%\end{equation}
%For $i=2$ we apply the general three-point stencil scheme (\ref{eq-num-lele1d}), but only with  a forth-order accuracy since necessarily $\beta=c=b=0$:
%\begin{equation}
%\beta=0,\quad  c=0,\quad  \alpha=\frac{1}{4},\quad  a=\frac{1}{4},\quad  b=0.
%\end{equation}
%
%
%For the second derivative, the same procedure is applied. The general formula is:
%\begin{eqnarray}
%\beta f^{''}_{i-2}+\alpha f^{''}_{i-1}+f^{'}_{i}+\alpha f^{''}_{i+1}+\beta f^{''}_{i+2}=\\
%=c\frac{f_{i+3}-2f_{i}+f_{i-3}}{9h^{2}}+b\frac{f_{i+2}-2f_{i}+f_{i-2}}{4h^{2}}+a\frac{f_{i+1}-2f_{i}+f_{i-1}}{h^{2}}
%\label{eq-num-lele2d}
%\end{eqnarray}
%with the corresponding linear system for the coefficients for a sixth-order truncation error:
%\begin{eqnarray}
%a+b+c=1+2\alpha+2\beta    \\
%a+2^{2}b+3^{2}c=\frac{4!}{2!}\left(\alpha+2^{2}\beta \right)  \\
%a+2^{4}b+3^{4}c=\frac{6!}{4!}\left(\alpha+2^{4}\beta \right)
%\end{eqnarray}
%We also considered a three-point stencil scheme for the second derivative, defined by:
%\begin{equation}
%\beta=0,\quad   c=0,\quad   \alpha=\frac{2}{11},\quad   a=\frac{12}{11},\quad   b=\frac{3}{11},
%\end{equation}
%and truncation error $T=-\frac{8\cdot 23}{11 \cdot 8!} h^{6} f^{(8)}$.
%
%For the boundary node $i=2$ we use the same scheme (\ref{eq-num-lele2d}) but with forth-order accuracy:
%\begin{equation}
%\beta=0,\quad  c=0,\quad  \alpha=\frac{1}{10},\quad  a=\frac{6}{5},\quad  b=0,
%\end{equation}
%and for $i=1$ we use a scheme of the form
%\begin{equation}
%f^{''}_{1}+\alpha f^{''}_{2}=\frac{1}{h^{2}}\left( af_{1}+bf_{2}+cf_{3}+df_{4} \right)
%\end{equation}
%with the following coefficients for a third-order accuracy:
%\begin{equation}
%  \alpha=11,\quad   a=13,\quad   b=-27,\quad   c=15,\quad  d=-1.
%\end{equation}
%
%It is important to note the chosen schemes use a three-point stencil (even for boundary nodes) and imply the resolution of a linear system with tridiagonal matrix. This is done in a very efficient matter using the Thomas algorithm (a version of the LU decomposition) adapted to compute the derivatives for the nodes of an entire plane $(y,z)$. The additional time necessary to compute the derivatives is thus very low.
%
%
%
%Since we use a staggered grid, an high-order interpolation procedure is necessary. The principle is the same and described by \cite{Boersma-2005}. We have chosen method which has the same formal accuracy as the one for the derivatives. The compact interpolation rule is the following: 
%\begin{equation}
%f_i + a(f_{i+1}+f_{i-1})=b(f_{i+1/2} + f_{i-1/2})+ c(f_{i+3/2}+f_{i-3/2})+ d(f_{i+5/2}+f_{i-5/2}) + e(f_{i+7/2} +f_{i-7/2}).
%\label{interp_gen}
%\end{equation}
%
%The values of the coefficients a, b, c, d, e for the sixth order accuracy are:
%\begin{equation}
%a=3/10, b=3/4, c=1/20, d=0, e=0.
%\end{equation}
%Closer to the boundary the stencil is smaller and a fourth order interpolation is obtained for: 
%\begin{equation}
%a=1/6, b=2/3.
%\label{interp_o4}
%\end{equation}
%At the boundary a third order accurate extrapolation is used for the first grid point:
%\begin{equation} 
%f_i= \frac{15}{8} f_{i+1/2} - \frac{5}{4} f_{i+3/2} + \frac{3}{8} f_{i+5/2}.
%\end{equation}
%Formulas  (\ref{interp_gen}) to (\ref{interp_o4}) form a tridiagonal system which is solved using the Thomas algorithm. 
%
%The same interpolation rules are used for the interpolation following the perpendicular direction, by shifting the data over a half grid cell from $f_i$ to $f_{i+1/2}$, obtaining $n-1$ points instead of $n$. 
%At the borders, the interpolation scheme is  \citep{Nagarajan-2003}:
%\begin{equation}
%f_{1/2} =a' f_{1} + b' f_{2}+ c' f_{3}+ d' f_{4},
%\end{equation}
%and a forth order accuracy formulation is obtained for:
%\begin{equation}
%a'=5/16, b'=15/16, c'=5/16, d'=1/16.
%\end{equation}
%
%The discretization of the momentum equations using compact schemes on a staggered grid will follow a different procedure than for second-order schemes. We consider the same illustration of this calculation, using the momentum equation following the $y$ direction. The convective terms are rewritten here in a non-conservative form.  
%\begin{figure}[!htbp]
%\begin{minipage}{1.\linewidth}
%\begin{center}
% {\includegraphics[width=0.3\textwidth]{cap_3/interp_2.jpg}}
%  \end{center}
%  \end{minipage}
%   \caption{Illustration of the interpolation scheme on the staggered grid: following the y-direction, the velocity $w$ must be interpolated at point $D$.}
%    \label{fig-interpy}
%  \end{figure}
%
%Following fig. \ref{fig-interpy}, the velocities $u$ and $w$ are first interpolated at point $D$. Each term  is then obtained by multiplying the interpolated velocity (obtained with \ref{interp_gen}) and the sixth order derivative (\ref{eq-num-lele1d}) as follows:
%
%$\bullet$ {Convective terms following the y-direction} computed at $(i+1/2,j,k+1/2)$\\
%\begin{eqnarray}
%\nonumber
%v \frac{\partial v}{\partial y}= v(i,j,k)~~~~ \frac{\partial v}{\partial y}(i,j,k),\\
%\nonumber
%u \frac{\partial v}{\partial x}=u(i,j,k)|_{D}~\frac{\partial v}{\partial x}(i,j,k),\\
%\nonumber
%w \frac{\partial v}{\partial z}=w(i,j,k)|_{D}~\frac{\partial v}{\partial z}(i,j,k).
%\nonumber
% \end{eqnarray} 
%
%The diffusive terms at $(i+1/2,j,k+1/2)$ are obtained simply by applying the sixth-order scheme (\ref{eq-num-lele2d}) (the interpolation is not needed) as:
% \begin{eqnarray}
% \nonumber
% \frac{\partial^{2} v}{\partial y^{2}}=\frac{\partial^{2} v}{\partial y^{2}}(i,j,k),\\
% \nonumber
%\frac{\partial^{2} v}{\partial x^{2}}=\frac{\partial^{2} v}{\partial x^{2}}(i,j,k),\\
%\nonumber
%\frac{\partial^{2} v}{\partial z^{2} }=\frac{\partial^{2} v}{\partial z^{2} }(i,j,k).
%\nonumber
% \end{eqnarray}
%
%
%
%\subsection{Discrete formulation of the temperature equation: TVD scheme}
%
%
%The use of the centered second order finite difference scheme may introduce oscillations when discontinuities are present.
%The concept of a total variation diminishing method was introduced by \cite{Harten-1983} and allows the capture of steep temperature gradients, while avoiding localized oscillations. The TVD scheme preserves the monotonicity of the numerical solution. This propriety ensures that the local augmentation of a gradient during the time integration is compensated by the diminishing of a gradient at a different location within the domain.  
%
%
%In our case, TVD schemes are necessary to treat convective terms in the temperature equation (\ref{eq-num-final3}) in order to keep the values of the temperature between a minimum and a maximal value fixed by the initial conditions.
%In Cartesian coordinates this equation reads:
%\begin{equation}
%\frac{\partial  T}{\partial t}+\frac{\partial  T u}{\partial x}+\frac{\partial  T v}{\partial y}+\frac{\partial  T w}{\partial z}=\frac{1}{\Rey \Prd}
%\left(\frac{\partial^{2} T}{\partial x^{2}}+\frac{\partial^{2} T}{\partial y^{2}}+\frac{\partial^{2} T}{\partial z^{2}} \right)
%\end{equation}
%The convective terms are computed after a general scheme proposed by \cite{vreug-1993}:
%\begin{equation}
%  \frac{\partial  Tu}{\partial x} = \frac{\emph{F}_{i+\frac{1}{2}}-\emph{F}_{i-\frac{1}{2}}}{\delta x},
%\end{equation}
%where:\\
%- for $ u_{i+\frac{1}{2}}>0$ the flux at the face $i+\frac{1}{2}$ is:
%\begin{eqnarray}
%\emph{F}_{i+\frac{1}{2}}=\left[   T_{i}+\frac{1}{2}\Phi (c_{i+\frac{1}{2}})( T_{i}- T_{i-1}) \right] u_{i+\frac{1}{2}},\\
%c_{i+\frac{1}{2}}=\frac{ T_{i+1}- T_{i}+\varepsilon}{ T_{i}- T_{i-1}+\varepsilon},
%\end{eqnarray}
%- for $ u_{i+\frac{1}{2}}<0$ the same flux is computed as:
%\begin{eqnarray}
%\emph{F}_{i+\frac{1}{2}}=\left[   T_{i}+\frac{1}{2}\Phi (c_{i+\frac{1}{2}})( T_{i+1}- T_{i+2}) \right] u_{i+\frac{1}{2}},\\
%c_{i+\frac{1}{2}}=\frac{ T_{i}- T_{i+1}+\varepsilon}{ T_{i+1}- T_{i+2}+\varepsilon},
%\end{eqnarray}
%where $\varepsilon=10^{-11}$ and the limiter:
%\begin{equation}
%  \Phi(c)=max\left[0,min\left( 2c,min\left(\frac{1}{3}+\frac{2}{3}c,2 \right) \right) \right]
%\end{equation}
% The diffusive terms in the temperature equation are treated similarly to the momentum equation.
%
%
%\subsection{Boundary conditions}
%% % % % % % % % % % % % % % % % % % % % % % % %
%
%
%For simulating flows inside a cavity, non-slip wall (zero velocity) boundary conditions have to be imposed.
%We made use of ghost cells which are defined as an additional cells beyond the border (with the index $0$ and $N+1$).
%The velocities inside the ghost cell are computed by linear interpolation between the first point inside the domain and the zero value imposed on the boundary. If we consider the same illustration following the $y$ direction 
%the non-slip wall conditions were imposed as follows:
%\begin{eqnarray}
%\nonumber
%u(i,0,k)&=&-u(i,1,k),\\
%\nonumber
%w(i,0,k)&=&-w(i,1,k),\\
%\nonumber
% \\
% \nonumber
%u(i,n2,k)&=&-u(i,n2-1,k),\\
%w(i,n2,k)&=&-w(i,n2-1,k),\\
%\nonumber
%\\
%\nonumber
%v(i,0,k)&=&-v(i,2,k),\\
%\nonumber
%v(i,1,k)&=&0,\\
%\nonumber
%v(i,n2,k)&=&0.
%\end{eqnarray}
%The same procedure is applied for the intermediate velocities of the predictor step, since we impose $\widetilde{u}=\widetilde{v}=\widetilde{w}=0$ on the boundary of the computational domain. 
%
%The same ghost cells are used to impose the boundary conditions for the temperature. We may consider Dirichlet or Neumann boundary conditions. For the same direction $y$, if\\
%$T=T_h$ for $y=0$ and $T=T_c$ for $y=L_y$ (Dirichlet conditions for the Rayleigh-B{\'e}nard case), then
%\begin{eqnarray}
%\nonumber
%T(i,0,k)&=& 2 T_h - T(i,1,k),\\
%\nonumber
%T(i,n2,k)&=&2 T_c - T(i,n2-1,k),
%\end{eqnarray}
%and if  $\partial T/\partial n =0$ for both $y=0$ and $y=L_y$ (Neumann conditions for the temperature driven cavity), then
%\begin{eqnarray}
%\nonumber
%T(i,0,k)&=& T(i,1,k),\\
%\nonumber
%T(i,n2,k)&=& T(i,n2-1,k).
%\end{eqnarray}
%
%
%Finally, we recall that for solving the Poisson equation (\ref{eq-num-phi}), Neumann boundary conditions are applied everywhere to ensure the compatibility with the boundary conditions on the velocity. The procedure for solving the Poisson equation takes into account these conditions implicitly, without using the ghost cells.
%
%
%\pagebreak
%
%\section{Structure of the Fortran90 simulation code}
%
%
%The numerical method presented below was implemented in Fortran 90. The structure of the code is presented in figure \ref{fig-codestruct} showing the main modules.
%\begin{figure}[!htbp]
%\begin{minipage}{1.\linewidth}
%\begin{center}
% {\includegraphics[width=1.\textwidth]{cap_3/code_structure.JPG}}
% \caption{Navier-Stokes-Boussinesq numerical code: general structure with modules in Fortran 90.}
%  \end{center}
%  \end{minipage}
%    \label{fig-codestruct}
%  \end{figure}
%
%The code is structured in independent modules, with dynamic allocation of arrays, which make it very flexible in setting computational configurations and/or different mathematical models. The main parameters of the simulation are set by the user via a $\textbf{general menu}$; it is possible to define values for:\\
%$\bullet$ domain dimensions, number of grid points,\\
%$\bullet$ Reynolds, Rayleigh, Prandtl numbers,\\
%$\bullet$ type of mesh for each direction (uniform or stretched),\\
%$\bullet$ type of boundary conditions for velocity (periodic or slip-wall or non-slip wall),\\
%$\bullet$ type of boundary conditions for temperature (Dirichlet or Neumann) and their localization,\\
%$\bullet$ time integration scheme (Euler, Adams-Bashforth, Runge-Kutta),\\
%$\bullet$ time integration parameters (final time, time step, etc),\\
%$\bullet$ save and restart options with prescribed names for the files, etc.
%
%
%In the main program there are several steps followed as:
%
%(1) The initialization phase:  a lecture of the general menu is made and all the parameters of the simulation set. The dynamic allocation of the variables is also done in order to save time and computer memory. The mesh is generated by calling the mesh module. Further the initial flow field values are set, together with boundary conditions imposed for the velocity, temperature and pressure field. The Boussinesq gravity term is also computed and an initialization of the Poisson solver is done before the time loop.  
%
%(2) The main time advancement loop: the main loop begins by saving the previous velocity and temperature values, at previous time step.  The time advancement step is adapted to  the type of integration scheme: one-step first-order Euler, one-step second Adams-Bashforth scheme and three-step Runge-Kutta scheme. Within this loop : the boundary conditions are imposed; the gravity term computed; the momentum equation, Poisson equation and passive scalar evolution equation are solved. As discussed in the previous section, after the momentum equation is solved a correction is made  by solving the Poisson equation. The Poisson solver is implemented into an independent module and make use of FFT libraries and FISHPACK. 
%
%After the time loop is finished the total divergence of the velocity filed is computed, the values of the flow filed are update. When a stationary state is searched, we monitor the Euclidean norm of the variation of velocities and temperature from one step to another.
%
%(3) Final phase : save files with flow field variables and print the computing time.
%
%
%The $\textbf{main program}$ makes use of several modules. Each module is specific to a certain task and is called when needed. We give here a short description of the role of each module. 
%
%The $\textbf{menu}$ module reads the values from the general menu, discussed before, and creates an echo of the parameters.
%
%The $\textbf{flow variables}$ module allocates each vector to the estimated size used in the computations. 
%
%The $\textbf{mesh}$ module, generates the uniform or variable grids; different subroutines implementing transform coordinates for stretched meshes are present; the different metrics are also computed.  
%
%The $\textbf{initial flow}$ module initialize the flow field variables (from a prescribed initial condition or from a restart file). 
%
%The $\textbf{boundary conditions}$ module contains several subroutines, each with a particular task:\\
%$\bullet$ the first one is used to impose the velocity boundary conditions; for each direction the boundary values are evaluated separately;\\
%$\bullet$ the second one imposes the Dirichlet type boundary conditions and it is called specifically for the direction where we wish to enforce these values;\\
%$\bullet$ the third one is used to set the values of variables of ghost cells, as function of the adjacent cells inside the flow field;\\
%$\bullet$ the fourth subroutine evaluates the boundary conditions for the pressure for all three directions.
%
%The $\textbf{boussinesq force}$ module evaluates the value of the body-force terms for each direction. 
%In our applications, only the gravity is present and  we make a single call of this subroutine, following the z-direction.
%
%The $\textbf{momentum equation}$  the momentum equation using either  second order or sixth order finite difference schemes. 
%
%The $\textbf{Poisson}$ module solves  the projection equation at each time step. It has different subroutines for 2D and 3D cases as well as the possibility to change between a periodical flow after the $x$ direction (Fourier series development) and a non-periodical case (cosine series).
%
%The $\textbf{passive scalar}$ module computes the explicit terms of the temperature equation using either a  second order scheme, or the TVD scheme or a sixth order compact scheme, depending on the initial choice.
%
%The $\textbf{output}$ module is set to produce either binary files for post-processing or directly files adapted for a Tecplot360 visualization of the flow field variables.  It produces three-dimensional files as  well as two dimensional sections within the cavity. 
%
%
%%\pagebreak
%
%\section{Study of the order of the numerical method}
%% % % % % % % % % % % % % % % % % % % % % % % % % % % % % % %
%
%Compact schemes are derived to formally obtain low truncation errors, ${cal O}(h^p)$, with $h$ the grid size and $p$ the order of the scheme. However, when non-periodic domains are used, the order of the method has to be decreased at the boundaries in order to maintain the band-structure of the matrix defining the scheme. For the sixth-order scheme used here, we formally have $p=6$ for inner points, and near the boundaries $p=4$ for $i=2$ (or $i=N-1$) and $p=3$ for $i=1$ (or $i=N$). Since we deal with an implicit scheme, all the linear equations giving the values of the derivatives are coupled and the global accuracy of the scheme is affected by the order decrease near the boundaries. 
%
%It was emphasized in previous studies \citep{S.K.Lele-1992} that the most appealing feature of compact schemes is the accurate representation of large range of wave numbers, rather than the resolution of a single wave. Nevertheless, we consider that it could be useful to estimate the global accuracy of the scheme when non-periodic boundaries are used. For this purpose, we undertake in this section some simple academic tests to measure the overall order of the derivation scheme: the derivation of a given analytical function (1D test) and the computation of the 2D Burggraf flow, which is an analytical, manufactured solution of the Naviers-Stokes equations. For both cases, the exact solution is known and could be used to compute the order of the method using the following classical indicators:
%\begin{itemize}
%\item[$\bullet$] Estimation of the approximation error: if the exact solution $\Phi$ is represented on the grid $h$ by the numerical solution $\phi_h$ as
%\begin{equation}
%\Phi = \phi_h + C\, h^p +{\cal O}(h^{p+1}),
%\label{eq-num-errh}
%\end{equation}
%we can compute the approximation (or discretization) error $\varepsilon_h= \Phi - \phi_h$. We can measure $\varepsilon_h$ for a given point of the computational domain (local error estimator) or compute various norms $\|\varepsilon_h\|$ (global error estimator). We use the following general definition of functional norms:
%\begin{equation}
%  \label{eq-num-Normes}
%  \begin{array}[h]{lcl}\vspace{0.2cm}
%\displaystyle \|v(x)\|_{\infty} =\sup_x \, |v(x)| &
%\Longrightarrow & \displaystyle \|V\|_\infty =  \sup_i \, |V_i|, \\
%\vspace{0.2cm} \displaystyle \|v(x)\|_{1} = \int_0^L \, |v(x)| dx
%& \Longrightarrow & \displaystyle
%\|V\|_1 = h \sum _{i=1}^N \,|V_i|, \\
%\displaystyle \|v(x)\|_{2} =\left[\int_{0}^{L}\, v^2(x)
%dx\right]^{1/2} & \Longrightarrow & \displaystyle \|V\|_2 =
%h^{1/2} \left[\sum _{i=1}^N \, (V_i)^2\right]^{1/2}.
%  \end{array}
%\end{equation}
%By computing $\|\varepsilon_h\|$ for different fine grids, we can represent $\|\varepsilon_h(h)\|$ in logarithmic coordinates and measure the slope of this curve, which gives an estimation of $p$.
%
%
%\item[$\bullet$] Estimation of the order of convergence by Richardson method: if (\ref{eq-num-errh}) is also written for two other coarser grids $(2h)$ and  $(4h)$
%\begin{eqnarray}
%\Phi&=& \phi_{4h} + C\, (4h)^{p}+{\cal O}(h^{p+1}),\\ \label{eq-num-err2h}
%\Phi&=& \phi_{2h} + C\, (2h)^{p}+{\cal O}(h^{p+1}),
%\end{eqnarray}
%we obtain, by neglecting higher orders than $p$, 
%\begin{equation}
% p= \frac{\displaystyle \ln \left(\frac{\phi_{2h}-\phi_{4h}}{\phi_{h}-\phi_{2h}}\right) }{\ln 2}.
% \label{eq-num-Richard}
%\end{equation}
%From (\ref{eq-num-errh}) and (\ref{eq-num-err2h}), we can estimate the approximation (discretization) error:
%\begin{equation}
% \varepsilon_h=\Phi - \phi_h = \displaystyle  \frac{\phi_{h}-\phi_{2h}}{2^p - 1},
%\end{equation}
%and use $\phi_h+\varepsilon_h$ as a better estimation of the solution $\Phi$. This is the Richardson extrapolation method - it implicitly assumes that the discretization error is monotonically decreasing and the grid is sufficiently refined. This method is often used as a tool for code validation, or an indicator of the grid convergence (local estimation of $p$ could provide indication of the need of mesh refinement in that zone).
%\end{itemize}
%
%
%\subsection{One dimensional case: derivation of an analytical function}
%% % % % % % % % % % % % % % % % % % % % % % % % % % % % % % % % % % % % % % % %
%
%We first consider a single wave function over a domain $L=2\pi$ :
%\begin{equation}
%f(x)= \sin(ax),\quad
%f'(x)=a \cos(x),\quad
%f''(x)=-a^2 \sin(ax).
%\end{equation}
%and use second and sixth order finite difference schemes to compute the first and second derivative. At the boundaries ($i=1$ and $i=N$), the centered second order scheme degenerates into a first order scheme.
%Figure \ref{fig-num-sin-eps} displays the approximation error $\varepsilon_h$.
%\begin{figure}[!htbp]
%\begin{minipage}{0.4\linewidth}
%\begin{center}
% {\includegraphics[width=\textwidth]{cap_2B/t1D_funcTRIG_eps_dudx_iper1_o2fort.pdf}}
%\end{center}
%\end{minipage}
%{\small{second order}}
%\begin{minipage}{0.4\linewidth}
%\begin{center}
% {\includegraphics[width=\textwidth]{cap_2B/t1D_funcTRIG_eps_d2udx2_iper1_o2fort.pdf}}
%\end{center}
%\end{minipage}\\
%\begin{minipage}{0.4\linewidth}
%\begin{center}
% {\includegraphics[width=\textwidth]{cap_2B/t1D_funcTRIG_eps_dudx_iper1_o6fort.pdf}}
%\end{center}
%\end{minipage} {\small sixth order}
%\begin{minipage}{0.4\linewidth}
%\begin{center}
% {\includegraphics[width=\textwidth]{cap_2B/t1D_funcTRIG_eps_d2udx2_iper1_o6fort.pdf}}
%\end{center}
%\end{minipage}
%\caption{Derivation of a single wave (sine) function. Approximation error $\varepsilon_h$ for $N=512$ discretization points.}
%\label{fig-num-sin-eps}
%\end{figure}
%
%We first note that the discretization error for the sixth-order scheme is several orders of magnitude lower than for the second order method. However, for both second and sixth order approximations, we can see the influence of the degeneracy of the accuracy of the scheme near the boundaries. We also notice that $\varepsilon_h / h{^p}$ is not constant, which makes questionable the assumption (\ref{eq-num-errh}) ($C$ depends in fact on some derivative of the function). These observation suggests that global estimations of the order of accuracy (based on norm or Richardson analysis) has to be considered very carefully.
%
%
%The global error estimation in fig. \ref{fig-num-sin-errG} is based on different norms. The slope of the curves $\|\varepsilon\| (h)$ is directly computed by a linear least-squares method. 
%As expected, for the second order method the influence of the first grid point is not visible and $p=2$ is obtained for both first and second derivative.  For the sixth-order method, the influence of the border points is expected to bee more important, and, indeed, an order $p$ between 3 and 6 is obtained, depending on the norm considered. The particular form of the function $f$ makes the order $p$ different for the first and second derivatives (super-convergence of the first derivative). We have checked that by changing $f$ to a cosine function instead of the considered sine function, the results for the first and second derivative are symmetrically switched. 
%\begin{figure}[!htbp]
%\begin{minipage}{0.4\linewidth}
%\begin{center}
% {\includegraphics[width=\textwidth]{cap_2B/t1D_funcTRIG_errG_dudx_iper1_o2fort.pdf}}
%\end{center}
%\end{minipage}
%{\small second order}
%\begin{minipage}{0.4\linewidth}
%\begin{center}
% {\includegraphics[width=\textwidth]{cap_2B/t1D_funcTRIG_errG_d2udx2_iper1_o2fort.pdf}}
%\end{center}
%\end{minipage}\\
%\begin{minipage}{0.4\linewidth}
%\begin{center}
% {\includegraphics[width=\textwidth]{cap_2B/t1D_funcTRIG_errG_dudx_iper1_o6fort.pdf}}
%\end{center}
%\end{minipage} {\small sixth order}
%\begin{minipage}{0.4\linewidth}
%\begin{center}
% {\includegraphics[width=\textwidth]{cap_2B/t1D_funcTRIG_errG_d2udx2_iper1_o6fort.pdf}}
%\end{center}
%\end{minipage}
%\caption{Derivation of a single wave (sine) function. Global error estimation plotting $\|\varepsilon_h\|$ versus the grid size $h$. The order $p$ of the method is computed for different norms using a linear least-squares method.}
%\label{fig-num-sin-errG}
%\end{figure}
%
%We look now at the local error estimator, considering the first grid point ($x=0$) and the middle point ($x=L/2$). The modulus of the discretization error $\varepsilon$ for these points is represented in fig. \ref{fig-num-sin-errL}. For the second order method, $p=2$ is again obtained for both points. For the second derivative, we note a super-convergence effect for the first point. Also, the order $p$ of the second derivative could not be estimated for the middle point, since the discretization errors are very low, ranging in the round-off error domain; this is also the case for the sixth-order method and is due to the particular form of the function ($d^2f/dx^2(L/2)=0$). 
%
%For the sixth-order method, we focus on the first derivative and notice that for the middle point the theoretical value $p=6$ is obtained. For the first point, $p=4$ shows a super-convergence for the first derivative, while the theoretical $p=3$ is obtained for the second derivative. We notice again the much lower orders of magnitude of the discretization error when sixth-order schemes are used.
%
%As a last remark for this case of derivation of a single wave (sine) function, we mention that the same study using periodic boundary conditions showed that the theoretical order $p=2$ for the second order method and $p=6$ for the sixth-order scheme are exactly found from computations.
%\begin{figure}[!htbp]
%\begin{minipage}{0.4\linewidth}
%\begin{center}
% {\includegraphics[width=\textwidth]{cap_2B/t1D_funcTRIG_errL_dudx_iper1_o2fort.pdf}}
%\end{center}
%\end{minipage}
%{\small second order}
%\begin{minipage}{0.4\linewidth}
%\begin{center}
% {\includegraphics[width=\textwidth]{cap_2B/t1D_funcTRIG_errL_d2udx2_iper1_o2fort.pdf}}
%\end{center}
%\end{minipage}
%\\
%\begin{minipage}{0.4\linewidth}
%\begin{center}
% {\includegraphics[width=\textwidth]{cap_2B/t1D_funcTRIG_errL_dudx_iper1_o6fort.pdf}}
%\end{center}
%\end{minipage} {\small sixth order}
%\begin{minipage}{0.4\linewidth}
%\begin{center}
% {\includegraphics[width=\textwidth]{cap_2B/t1D_funcTRIG_errL_d2udx2_iper1_o6fort.pdf}}
%\end{center}
%\end{minipage}
%\caption{Derivation of a single wave (sine) function. Local error estimation plotting $\|\varepsilon_h\|$ versus the grid size $h$ for fixed points ($x=0$ and $x=L/2$). The order $p$ of the method is computed for different norms using a linear least-squares regression.}
%\label{fig-num-sin-errL}
%\end{figure}
%
%
%
%For the same exercise of the derivation of an analytical function, we take a second case considering a $\textbf{polynomial function}$ from \cite{Lamballais-2009}:
%\begin{equation}
%f(x)=\frac{x^7}{7}-\frac{x^6}{2}+\frac{17x^5}{25}-\frac{9x^4}{20}+\frac{274x^3}{1875}-\frac{12x^2}{625}
%\end{equation}
%The function is neither periodical nor symmetrical. Its boundary conditions are: $ f'(0)= 0 $ and $f'(L)= 0$. 
%The computations of its first and second derivative can be compared with exact values:
%\begin{equation}
%f'(x)=(x-1)\left(x-\frac{4}{5}\right)\left(x-\frac{3}{5}\right)\left(x-\frac{2}{5}\right)\left(x-\frac{1}{5}\right)x,
%\end{equation}
%
%\begin{equation}
%f''(x)= 6x^5 - 15x^4 + \frac{68}{5} x^3 -\frac{27}{5} x^2 +\frac{548}{625} x - \frac{24}{625}.  
%\end{equation}
%
%%Figure \ref{fig-num-poly-eps} displays the approximation error $\varepsilon_h$. 
%%\begin{figure}[!htbp]
%%\begin{minipage}{0.4\linewidth}
%%\begin{center}
%% {\includegraphics[width=\textwidth]{cap_2B/t1D_funcPOLY_eps_dudx_iper1_o2fort.pdf}}
%%\end{center}
%%\end{minipage}
%%{\small second order}
%%\begin{minipage}{0.4\linewidth}
%%\begin{center}
%% {\includegraphics[width=\textwidth]{cap_2B/t1D_funcPOLY_eps_d2udx2_iper1_o2fort.pdf}}
%%\end{center}
%%\end{minipage}\\
%%\begin{minipage}{0.4\linewidth}
%%\begin{center}
%% {\includegraphics[width=\textwidth]{cap_2B/t1D_funcPOLY_eps_dudx_iper1_o6fort.pdf}}
%%\end{center}
%%\end{minipage} {\small sixth order}
%%\begin{minipage}{0.4\linewidth}
%%\begin{center}
%% {\includegraphics[width=\textwidth]{cap_2B/t1D_funcPOLY_eps_d2udx2_iper1_o6fort.pdf}}
%%\end{center}
%%\end{minipage}
%%\caption{Derivation of a polynomial function. Approximation error $\varepsilon_h$ for $N=512$ discretization points.}
%%\label{fig-num-poly-eps}
%%\end{figure}
%
%Global error estimation in displayed in fig. \ref{fig-num-poly-errG}. We  notice again the boundary effects and the lower order of the discretization error when sixth-order schemes are used. The boundary discretization effects do  not affect the global precision of the second order method; in exchange, they are critical for the sixth-order schemes since the global order is reduced to $p=3$ (the discretization order of the first and last points). 
%\begin{figure}[!htbp]
%\begin{minipage}{0.4\linewidth}
%\begin{center}
% {\includegraphics[width=\textwidth]{cap_2B/t1D_funcPOLY_errG_dudx_iper1_o2fort.pdf}}
%\end{center}
%\end{minipage}
%{\small second order}
%\begin{minipage}{0.4\linewidth}
%\begin{center}
% {\includegraphics[width=\textwidth]{cap_2B/t1D_funcPOLY_errG_d2udx2_iper1_o2fort.pdf}}
%\end{center}
%\end{minipage}\\
%\begin{minipage}{0.4\linewidth}
%\begin{center}
% {\includegraphics[width=\textwidth]{cap_2B/t1D_funcPOLY_errG_dudx_iper1_o6fort.pdf}}
%\end{center}
%\end{minipage} {\small sixth order}
%\begin{minipage}{0.4\linewidth}
%\begin{center}
% {\includegraphics[width=\textwidth]{cap_2B/t1D_funcPOLY_errG_d2udx2_iper1_o6fort.pdf}}
%\end{center}
%\end{minipage}
%\caption{Derivation of a polynomial function. Global error estimation plotting $\|\varepsilon_h\|$ versus the grid size $h$. The order $p$ of the method is computed for different norms using a linear least-squares method.}
%\label{fig-num-poly-errG}
%\end{figure}
%
%Since the global error estimator is not appropriate to assess for the features of the scheme, we look in fig. \ref{fig-num-poly-errL} at particular points  of the domain, the first point ($x=0$) and the middle point ($x=L$), as in \cite{Lamballais-2009}. For the first derivative, $p=2$ is obtained for the second order scheme and both derivatives. Again, for the middle point the discretization errors are very small for the second derivative and the order of both methods could not be determined accurately. For the sixth-order method,   we notice the expected $p=3$ for the first point and $p=6$ for the middle point when the first derivative is considered. Also, $p=3$ is obtained for the first point when the second derivative is computed. 
%
%It is important to emphasize that in our sixth-order scheme, the ghost cells are not used. In \cite{Lamballais-2009} it was reported that the use of ghost cells together with the sixth-order scheme can decrease dramatically the order of convergence: $p=2$ was obtained when a simple linear interpolation was used for the ghost cells. A special treatment of the ghost cells was proposed by \cite{Lamballais-2009} to restore the sixth-order convergence for inner points of the domain. In our case, we prefer to use the sixth-order scheme without ghost cells since the computational effort is exactly the same. 
%
%Finally, we note that the results for the polynomial function are consistent with those obtained for the trigonometrical one. Both function were chosen to respect the symmetrical conditions $f'=0$ for ($x=0$ and $x=L$). The accuracy of the sixth-order order is significantly improved compared to the second-order scheme when inner points are considered. That emphasizes the fact that a global error estimator is not always appropriate to characterize compact schemes since discretization errors are dominated by the boundary treatment. 
%\begin{figure}[!htbp]
%\begin{minipage}{0.4\linewidth}
%\begin{center}
% {\includegraphics[width=\textwidth]{cap_2B/t1D_funcPOLY_errL_dudx_iper1_o2fort.pdf}}
%\end{center}
%\end{minipage}
%{\small second order}
%\begin{minipage}{0.4\linewidth}
%\begin{center}
% {\includegraphics[width=\textwidth]{cap_2B/t1D_funcPOLY_errL_d2udx2_iper1_o2fort.pdf}}
%\end{center}
%\end{minipage}
%\\
%\begin{minipage}{0.4\linewidth}
%\begin{center}
% {\includegraphics[width=\textwidth]{cap_2B/t1D_funcPOLY_errL_dudx_iper1_o6fort.pdf}}
%\end{center}
%\end{minipage} {\small sixth order}
%\begin{minipage}{0.4\linewidth}
%\begin{center}
% {\includegraphics[width=\textwidth]{cap_2B/t1D_funcPOLY_errL_d2udx2_iper1_o6fort.pdf}}
%\end{center}
%\end{minipage}
%\caption{Derivation of a polynomial function. Local error estimation plotting $\|\varepsilon_h\|$ versus the grid size $h$ for fixed points ($x=0$ and $x=L/2$). The order $p$ of the method is computed for different norms using a linear least-squares regression.}
%\label{fig-num-poly-errL}
%\end{figure}
%
%
%Following this conclusion, another interesting question is how many points inside the computational domain will display a sixth order convergence rate? To answer this question, we consider the Richardson analysis (\ref{eq-num-Richard}) explained above. Figure \ref{fig-num-Richardson} shows the results for the first derivative only: the discretization error $\varepsilon$ is plotted first, followed by the computed values of $p$ for each grid point, using (\ref{eq-num-Richard}). For both second and sixth order schemes, the three meshes have $N=32, 64$ and $128$ points, respectively. When $\varepsilon < 10^{-12}$ we discard the calculated values of $p$, since  we are in the round-off errors domain and the method is not reliable. 
%
%For the second order method, all the grid points display the expected convergence rate $p=2$. For the sixth order scheme we notice that the middle zone of the domain ($0.4 < x < 0.6$) displays the theoretical convergence rate $p=6$. This zone corresponds to a monotonic decrease of the discretization error $\varepsilon$ with the grid resolution. Near the locations $x=L/4$ and $x=3L/4$, since $\varepsilon$ decreases not uniformly, the calculated value of $p$ exceed 6 and goes up to 15. This is explained by the faster reduction of the discretization error for $N=128$ (the plateau with low discretization errors is larger and larger when the resolution is increased). When going near the boundaries, the value of $p$ decreases again and reaches the theoretical value $p=3$. 
%
%This variation of the convergence order $p$, not seen in previous publications, which suggest that the Richardson method should be used with caution when compact implicit schemes are used.
%
%
%\begin{figure}[!htbp]
%\begin{minipage}{0.5\linewidth}
%\begin{center}
% {\includegraphics[width=1.\textwidth]{cap_2B/t1D_funcPOLY_Richardson_dudx_iper1_o2fort.pdf}}\\
%second order scheme
%\end{center}
%\end{minipage}
%\begin{minipage}{0.5\linewidth}
%\begin{center}
% {\includegraphics[width=1.\textwidth]{cap_2B/t1D_funcPOLY_Richardson_dudx_iper1_o6fort.pdf}}\\
%sixth-order scheme
%\end{center}
%\end{minipage}
%\caption{Derivation of a polynomial function (first derivative). Estimation of the convergence rate by the Richardson method. Discretization errors for the three grids considered (up) and calculated values of $p$ for the grid points of the coarse grid (down). }
%\label{fig-num-Richardson} 
%\end{figure}
%
%
%
%
%\subsection{Two-dimensional case: Burggraf flow}
%% % % % % % % % % % % % % % % % % % % % % % % % % % % % % % % %
%
%To test the accuracy of the Navier-Stokes solver we use the well-known \citep{Shih-1989,Sheu2004,Lamballais-2009} analytical solution called the Burggraf flow. It is in fact a manufactured solution, obtained by introducing an artificial force term in the momentum equations in order to get the prescribed solution. The Burggraf flow is a steady recirculating flow inside a square cavity $L_x=L_y=1$, with non-slip wall  conditions at the boundaries:
%\begin{equation}
%u(x,0)=u(0,y)=u(1,y)=0,
%\end{equation}
%excepting for the top boundary, where the flow is entrained by a horizontal velocity:
%\begin{equation}
%u(x,1)=16(x^4-2x^3+x^2).
%\end{equation}
%It should be noted that this form of the velocity avoids discontinuities at the corners of the cavity, which is the case for the classical lid driven flow. 
%
%
%The exact solution of this flow is
%\begin{eqnarray}
%u(x,y)&=&8(x^4-2x^3+x^2)(4y^3-2y),\\
%v(x,y)&=&-8(4x^3-6x^2+2x)(y^4-y^2),
%\end{eqnarray}
%and corresponds to a y-direction forcing:
%\begin{equation}
%f_y=\left( \frac{8}{Re} [24g+2g''h''+g^{(4)}h]+64 \left( \frac{g'^2}{2}(hh'''-h'h'')-hh'(g'g'''-g''^2) \right) \right) e_y, 
%\end{equation}
%with:
%\begin{equation}
%g(x)=\frac{x^5}{5}-\frac{x^4}{2}+\frac{x^3}{3}, \quad h(y)=y^4-y^2.
%\end{equation}
%
%
%We used the second order solver for $40$, $120$ and $360$ grid points, while for the sixth order we considered $20$, $60$ and $180$ grid points.  For the sixth-order a coarser grid offers the same accuracy as a more refined mesh for the second-order.  In both cases staggered mesh is considered and we recall that the sixth order interpolation was used with the compact scheme. The computations were done for a Rayleigh number $Ra=10^6$ and Prandtl number $Pr=0.71$.  
%
%
%Figure \ref{fig-o2B} shows the streamlines obtained by the second order solver. We also show the decrease of the global discretization error $\|\varepsilon\|_2$ with the grid size. The equivalent definition for a 2D domain of the $L^2$ norm (\ref{eq-num-Normes}) was used. The expected second order convergence is obtained for both horizontal and vertical velocities.
%\begin{figure}[!htbp]
%\begin{minipage}{0.25\linewidth}
%\begin{center}
% {\includegraphics[width=1.1\textwidth]{cap_2B/B_o2_str.pdf}}\\
%\end{center}
%\end{minipage}
%\begin{minipage}{0.35\linewidth}
%\begin{center} {\includegraphics[width=0.9\textwidth]{cap_2B/o2_solEx_errL_v_iper1_o2fort.pdf}}\\
%Error for vertical velocity
%\end{center}
%\end{minipage}
%\begin{minipage}{0.35\linewidth}
%\begin{center}
% {\includegraphics[width=0.9\textwidth]{cap_2B/o2_solEx_errL_w_iper1_o2fort.pdf}}\\
%Error for horizontal velocity
%\end{center}
%\end{minipage}
%\caption{Burggraf flow computed with the second order solver. Streamlines and global discretization error  $\|\varepsilon\|_2$ for the vertical and horizontal velocities.}
%\label{fig-o2B}
%\end{figure}
%
%
%For the sixth-order method the streamlines and decrease of the global error $\|\varepsilon\|_2$ are presented in figure \ref{fig-o6B}. The results obtained state that despite the higher order used for velocities derivatives, only a second order ($p=2.5$) accuracy is obtained. This behaviour was expected from the previous 1D study showing that global error estimates are greatly affected by the lower order of the scheme near the boundaries. We also recall that the Poisson solver used a second order discretization of the derivatives. A similar result ($p=2$)  for the Burggraf flow was reported by \cite{Lamballais-2009}; they claimed that the low order accuracy of the Poisson solver was responsible for this result. We can see now that this is not the only cause, and the global error estimates when the sixth order scheme is used could be misleading.  We note in passing that a fully sixth order Poisson solver will allow to increase the global precision of the flow solver \citep{Boersma-2011}. 
%\begin{figure}[!htbp]
%\begin{minipage}{0.25\linewidth}
%\begin{center}
% {\includegraphics[width=1.1\textwidth]{cap_2B/B_o6_str.pdf}}\\
%Streamlines
%\end{center}
%\end{minipage}
%\begin{minipage}{0.35\linewidth}
%\begin{center}
% {\includegraphics[width=0.9\textwidth]{cap_2B/o6_solEx_errL_v_iper1_o6fort.pdf}}\\
%Error for vertical velocity
%\end{center}
%\end{minipage}
%\begin{minipage}{0.35\linewidth}
%\begin{center}
% {\includegraphics[width=0.9\textwidth]{cap_2B/o6_solEx_errL_w_iper1_o6fort.pdf}}\\
%Error for horizontal velocity
%\end{center}
%\end{minipage}
%\caption{Burggraf flow computed with the sixth order solver.  Streamlines and global discretization error  $\|\varepsilon\|_2$ for the vertical and horizontal velocities.}
%\label{fig-o6B}
%\end{figure}
%\vspace{3.cm}
%
%At a first glance, the global accuracy for velocity and pressure is only slightly better for the sixth-order method, than for the second order one. It is interesting to compare in the following the grid resolution necessary to obtain the same level of the discretization error. Figure \ref{fig-profilesvbc} presents the vertical and horizontal velocity profiles, following the centerlines. We note that the  accuracy obtained with the finner grid for the second order solver ($360 \times 360$) is attained by using the sixth-order solver with only  ($180 \times 180$) grid cells. 
%\begin{figure}[!htbp]
%\begin{minipage}{0.5\linewidth}
%\begin{center}
% {\includegraphics[width=0.85\textwidth]{cap_3/burggraf_SN.jpg}}\\
%\end{center}
%\end{minipage}
%\begin{minipage}{0.5\linewidth}
%\begin{center}
% {\includegraphics[width=0.85\textwidth]{cap_3/burggraf_WE.jpg}}\\
%\end{center}
%\end{minipage}
%\caption{Burggraf flow. Profiles following the centerline of the cavity for the vertical and horizontal velocities.}
%\label{fig-profilesvbc}
%\end{figure}
%
%Even though the number of computational nodes could be divided by 4 when sixth order schemes are used, the computation of the derivatives in implicit schemes requires addition CPU time. 
%From figure \ref{fig-convd} it can be noted that while the computational time for an individual time step is higher for the sixth order than for the second-order accuracy, the overall global convergence time is lower since the sixth-order scheme reaches convergence in fewer steps than the second order scheme. This provides faster computational times for the sixth order to reach the stationary solution, for cases where the same number of grid points were considered. 
%\begin{figure}[!htbp]
%\begin{minipage}{0.32\linewidth}
%\begin{center}
% {\includegraphics[width=1.\textwidth]{cap_3/1step_grid.jpg}}\\
%a)
%\end{center}
%\end{minipage}
%\begin{minipage}{0.32\linewidth}
%\begin{center}
% {\includegraphics[width=1.\textwidth]{cap_3/TS_grid.jpg}}\\
%b)
%\end{center}
%\end{minipage}
%\begin{minipage}{0.32\linewidth}
%\begin{center}
% {\includegraphics[width=1.\textwidth]{cap_3/CPU_grid.jpg}}\\
%c)
%\end{center}
%\end{minipage}
%\caption{Burggraf flow. Variation with grid resolution $N$ of a) the computational time for one time step, b) the total number of steps necessary for convergence to the steady state ; c) the total computational time to reach the stationary state.}
%\label{fig-convd}
%\end{figure}
%
%

