%========================================================================
%\newcommand*\dsk{/home/danaila/A_COMMON/RECHERCHE}
%\newcommand*\dsk{D:/A_COMMON/RECHERCHE}

\newcommand*\dsk{../..}

\newcommand*\figpath{\dsk/ALL_FIGS/figs_CONVERGENCE_2017}
\newcommand*\bibpath{\dsk/ALL_BIBTEX}
\newcommand*\stypath{\dsk/ALL_STYLE}
\newcommand*\incpath{.}
%========================================================================

%========================================================================
%\documentclass[twoside,fontsize=11pt,headsepline,french]{scrbook}
\documentclass[twoside,fontsize=12pt,headsepline,english]{scrbook}
%========================================================================

%========================================================================

\usepackage{graphicx}   %% envir graphique pour latex ou pdflatex
\usepackage[T1]{fontenc}     %% pour la cesure avec les accents
\usepackage[latin1]{inputenc} %% problemes d'accents
\usepackage[colorlinks=true, linkcolor=myred, citecolor=myred]{hyperref}         %% links dans le dvi

\usepackage{amsmath,amsfonts,amssymb}
\usepackage{mathrsfs}
\usepackage{amsthm}
\usepackage{pdfpages}	% % introduire plusieurs pages pdf
\usepackage{booktabs}	% % Tableau style

%\usepackage[sectionbib,gather]{chapterbib}       %%  pour la biblio locale (projet)
\usepackage{times}            %% pour pdf avec de jolie police
\usepackage{colordvi}        %% couleurs dans le dvi
%\usepackage[monochrome]{color}
\usepackage{color}
\usepackage{xcolor}
%\usepackage{amssymb,epic}
%\usepackage[french]{minitoc}   % mini-sommaires pour chaque chapitre
\usepackage{multimedia}       %% pour les films
\usepackage{array,multirow,makecell}  % pour fusionner les cellules dans un tableau
%========================================================================
\usepackage[english]{babel}
%\usepackage[frenchb]{babel}    %% style french
%------------------------------%% titres en francais
%\addto\captionsfrench{\renewcommand{\figurename}{\em Figure}%
%\renewcommand{\tablename}{\em Tableau}%
%\renewcommand{\chaptername}{\em }
%\renewcommand{\summaryname}{Sommaire}}
%========================================================================
%\usepackage{\stypath//floatflt}         %% figures dans les paragraphes -- il est nul
%=======================================biblio en anglais(non compat avec cite)
\usepackage[round]{natbib}
%\usepackage[]{natbib} 

%\bibliographystyle{plainnat}
%                     \cite{}     %citation normale
%                     \citet{}    %textuel
%                     \cite[texte avant][texte apres]{}
%                     \citep{}    %avec parenth{\`e}ses
%                     \citealt{}, \citealp{}    %sans aucune parenth{\`e}se
%                      \citeyear{},\citeauthor{}
%
%=======================================biblio en francais
%\bibliographystyle{\stypath//plainnat-io}
%\bibliographystyle{abbrvnat-fr} % biblio style francais
%\usepackage{\stypath//frbib}
%\bibliographystyle{\stypath//frcomplet}
%%%%%%%%%%========================================================================
%   --- couleurs
\definecolor{myred}{rgb}{0.8,0.1,0.1}

\def\textRed{\color{red}}
\def\textBlack{\color{black}}
\def\textBlue{\color{blue}}
\def\Blue#1{{\color{blue}{#1}}}
\def\Red#1{{\color{red}{#1}}}
\def\Darkred#1{{\color{red}{#1}}}
\def\Green#1{{\color{green}{#1}}}
\def\Magenta#1{{\color{magenta}{#1}}}
%========================================================================
%========================================================================
% 
\def\vv{{\vec v}}
\def\vva#1{\vec{#1}}
\def\ds{\displaystyle}
\def\pl{\partial}
\newcommand{\Rey}{{\cal R}e}
\newcommand{\Prd}{{\cal P}r}
\newcommand{\Ray}{{\cal R}a}
%\newcommand{\Gra}{{\cal G}r}
\newcommand{\Bou}{{\cal B}o}
\newcommand{\Ste}{{\cal S} te}
\newcommand{\vref}{{\small ref}}

\newcommand {\R} {{\mathbb R}}
\newcommand {\N} {{\mathbb N}}
\newcommand {\C} {{\mathbb C}}
\newcommand {\Z} {{\mathbb Z}}
\newcommand{\ie}{{\em i.\thinspace{}e. }}
\newcommand{\etal}{{\em et al. }}
\newcommand{\eg}{{\em e.\thinspace{}g. }}

\def\cels#1{#1\, ${^\circ}$C}
\def\celsm#1{#1\, {^\circ}\mbox{C}}

\newcommand{\bigO}[1]{\ensuremath{\mathop{}\mathopen{}O\mathopen{}\left(#1\right)}}
%========================================================================
%mise en page

\usepackage[hmargin={3cm,2.3cm},vmargin={3cm,3cm}]{geometry}

%\topmargin -1.54cm   %marge en haut a 2cm %% on monte de 1.65 cm car dvi2ps marche en letter 11in
%\oddsidemargin 0cm   %marge a 2 cm
%\evensidemargin 0cm  %marge a 2 cm
%\parindent 0.cm
%\parskip 0.3cm
%\headsep 0.5cm \topskip .5cm \footskip 1.5cm \headheight 1.0cm
%\textwidth  15cm \textheight 22cm

%-------- dimensions pour les cadres des exercices, theoremes, etc
\def\leftsymb{0.5cm}
\def\rightenv{\textwidth}

%\def\minileft{5.3cm}
%\def\miniright{10.5cm}
\def\onefig{\textwidth}
\def\onefigS{0.6\textwidth}
\def\onefigM{0.75\textwidth}
\def\onefigB{0.99\textwidth}
\def\twofig{0.5\textwidth}
\def\threefig{0.33\textwidth}

\newtheorem{thm} {Th�or�me} [section]
\newtheorem{defi}[thm] {D�finition}
\newtheorem{prop} [thm] {Proposition}
\newtheorem{rem}[thm] {Remarque}
\newtheorem{concl}[thm] {Conclusion}
%%%%%%%%%%%%%%%%%%%%%%%%%%%%%%%%%%%%%%%%%%%%%%%%%%%%%%%%%%%%%%%%%%%%%%%%%%%%%%%
%%%%%%%%%%%%%%%%%%%%%%%%%%%%%%%%%%%%%%%%%%%%%%%%%%%%%%%%%%%%%%%%%%%%%%%%%%%%%%%
\def\LOGO{\hbox{\includegraphics[scale=0.75]{\stypath/logo-lmrs125.jpg}}}
\font\fonteupmc=pagk at 10 true pt
\font\plutotgros=pagk at 12 true pt
\def\LAN{\plutotgros}
\def\upmc{\fonteupmc}

\def\entete{
\vbox {
\hbox to \hsize{%
\hskip -1 cm\vbox{\LOGO}\hskip 0.45 true cm \vbox{\hbox{\LAN
Laboratoire de math�matiques Raphael Salem} \vskip .2 true cm \hbox{\upmc
Universit� de Rouen} \vskip .5 true cm
\hbox{\upmc Avenue de l'Universit�, BP.12,
  76801 Saint-�tienne-du-Rouvray}}
 \hfill}
 {\vskip .9 true cm
\hbox{\hskip -1 cm} \vskip .1 true cm
\hbox{\hskip -1 cm}}}}

%%%%%%%%%%%%%%%%%%%%%%%%%%%%%%%%%%%%%%%%%%%%%%%%%%%%%%%%%%% 
%%%%%%%%%%%%%%%%%%%%%%%%%%%%%%%%%%%%%%%%%%%%%%%%%%%%%%%%%%%%%%%%%%%%%%%%%%%%%%%
%%%%%%%%%%%%%%%%%%%%%%%%%%%%%%%%%%%%%%%%%%%%%%%%%%%%%%%%%%%%%%%%%%%%%%%%%%%%%%%
\titlehead{\hspace{1.5cm}\entete}
\title{\vspace{-1cm}
	\huge\Magenta{
1) Accuracy of the finite-element method for phase-change systems.\\
2) Study and evaluation of the Carman-Kozeny model.}\\
%{\includegraphics[width=0.35\textwidth]{\figpath/figs_bose/lattice_3.jpg}}
}
\author{\Large \Blue{Lei Bian \& Aina Rakotondrandisa  \& Ionut Danaila} }

\subject{\vspace{-2cm} Preliminary report}
%\date{}
\publishers{\small
%\begin{flushleft}
%	\Blue{This document is intended to evolve by gradually adding contributions. Feel free to correct or add parts to the latex source file, eventually using colors.} \Red{red}, \Blue{blue}, \Green{green}, etc.\\
%	The purpose of this study is twofold:
%	\begin{enumerate}
%	\item Evaluate the accuracy in time and space of the finite-element solver for phase change systems.\\ Idea: manufactured solutions will be used to estimate the space accuracy (stationary Burggraf flow) and time-dependent manufactured solution (as in \cite{nourgaliev2016fully}) for the time accuracy.
%	
%	\item Try to find an equivalence between the two models used to take into account the solid phase inside a single-domain approach: the viscosity penalty used in \cite{dan-2014-JCP} and the well-know Carman-Kozeny model (\eg \citep{belhamadia2012}).\\ Idea: study the mathematical derivation of the  Carman-Kozeny model from the Darcy equations; study the behaviour of the CK model compared to the viscosity penalisation for simple models/configurations (melting of a PCM).
%
%\end{enumerate} 	 
%\end{flushleft}
}

\begin{document}
%%%%%%%%%%%%%%%%%%%%%%%%%%%%%%%%%%%%%%%%%%%%%%%%%%%%%%%%%%%%%%%%%%%%%%%%%%%%%%%%

%
%\renewcommand{\refname}{Bibliography of chapter }
%\renewcommand*\partformat    {\partname~\thepart: } % regle le pb titre partie
\renewcommand*\partformat    {\thepart } % regle le pb titre partie
%%%%%%%%%%%%%%%%%%%%%%%%%%%%%%%%%%%%%%%%%%%%%%%%%%%%%%%%%%%%%%%%%%%%%%%%%%%%%%%%
%%%%%%%%%%%%%%%%%%%%%%%%%%%%%%%%%%%%%%%%%%%%%%%% initialisation minitoc
%\doparttoc %Table of contents for parts
%\dopartlof List of figures for parts
%\dopartlot List of tables for parts
%\dominitoc %Table of contents for chapters
%\dominilof List of figures for chapters
%\dominilot List of tables for chapters
%\dosecttoc Table of contents for sections
%\dosectlof List of figures for sections
%\dosectlot List of tables for sections


%%%%%%%%%%%%%%%%%%%%%%%%%%%%%%%%%%%%%%%%%%%%%%%%%%%%%%%%%%%%%%%%%%%%%%%%%%%%%%%%
%\maketitle


%%%%%%%%%%%%%%%%%%%%%%%%%%%%%%%%%%%%%%%%%%%%%%%%%%%%%%%%minitoc commands
%\parttoc Table of contents for parts
%\partlof List of figures for parts
%\partlot List of tables for parts
%\minitoc Table of contents for chapters
%\minilof List of figures for chapters
%\minilot List of tables for chapters
%\secttoc Table of contents for sections
%\sectlof List of figures for sections
%\sectlot List of tables for sections
%%%%%%%%%%%%%%%%%%%%%%%%%%%%%%%%%%%%%%%%%%%%
%%%%%%%%%%%%%%%%%%don't forget if needed %%%%%%%%%%%%%%%%%%%%%%%%%%%%%%%%%%%%%
%\chapter[toc version]{title version}
%\chaptermark{head version}
%\section[toc version]{title version%
%              \sectionmark{head version}}
%\sectionmark{head version}
%%%%%%%%%%%%%%%%%%don't forget if needed %%%%%%%%%%%%%%%%%%%%%%%%%%%%%%%%%%%%%

\frontmatter


%Print title NOW
\maketitle%

%Disable page numbering
\pagestyle{empty}

%########################################################################
% Multilingual abstracts
%########################################################################



%########################################################################
% Contents
%########################################################################

%\strut\newpage
\small
\tableofcontents

\newpage

%%%%%%%%=====================================================
%\pagestyle{myheadings}
\pagestyle{headings}
\renewcommand*{\chaptermarkformat}{%
\chapapp~\thechapter\autodot\enskip}


%%%%%%%%=====================================================
\mainmatter
%%%%%%%%%%%%%%%%%%%%%%%%%%%Le%%%%%%%%%%%%%%

%%%%%%%%%%%%%%%%%%%%%%%%%%%%%%%%%%%%%%%%%%%%%%%
%\graphicspath{{\figpath/CARMAN-COZENY}}
%\include{chap00_accuracy}
%\graphicspath{{\figpath/CARMAN-COZENY}}
%\include{chap01_CK_2017}
%\graphicspath{{\figpath/CARMAN-COZENY}}
%\include{chap02_CK_2017}
%\include{appendix}

\section{Introduction}

\section{Governing equations}
\subsection{Existing method}
\subsection{Enthalpy method}
\subsection{Navier-Stokes with Boussinesq approximation}

\section{Numerical method}
\subsection{Finite Element method}
\subsection{Newton algorithm}
\subsection{Mesh adaptation}

\section{2D configurations}
\subsection{Natural convection of air and water}
\subsection{PCM melting}
\subsection{Melting and solidification cycle}
\subsection{Water Freezing}

\section{3D configurations using hpddm}
\subsection{Natural convection of air in a cube cavity}
\subsection{Natural convection of air with obstacle}
\subsection{Natural convection in an exothermic building}
\subsection{PCM melting}

\section{Complexe configurations}
\subsection{PCM melting heated with natural convection of air}
\subsection{Solidification involving dendritic growth}

\boxed{$$\theta = \theta_{co}$$}

%%%%%%%%%%%%%%%%%%%%%%%%%%%%%%%%%%%%%%%%%%%%%%%

%%%%%%%%=====================================================
%\part{Bibliography}
%\parttoc
%%%%%%%=====================================================
\addcontentsline{toc}{chapter}{Bibliography}

\bibliographystyle{\stypath/plainnat-io}
%\bibliographystyle{plain}
%\renewcommand{\bibname}{\em }
\bibliography{\bibpath/bib_definition,\bibpath/danaila_publis,\bibpath/bib-fem,\bibpath/bib-PCM,\bibpath/bib-PCM-2016}
\end{document}
